\section{Timetable}
I start working on this project in September 2016. 
The development of the tasks are presented in Figure~\ref{timetable}. We target three conferences and one journal paper. 

\begin{center}
\begin{figure}[h]
\label{timetable}
\centering
\scalebox{0.70}{
\begin{tikzpicture}
\begin{ganttchart}[
x unit=0.36cm,
y unit title=1.0cm,
y unit chart=1.5cm,
%vgrid,
hgrid,
inline,
]{1}{48}
\gantttitle{Research Project}{48} \\
\gantttitle{2016}{12} \gantttitle{2017}{12} \gantttitle{2018}{12} \gantttitle{2019}{12}\\

\ganttbar[bar height=.4]{Literature Review and Comp .Exam}{6}{24}\\
%\ganttbar[bar height=.4]{DGA courses \& CARI, EPCARD}{6}{12}\\
\ganttmilestone[]{CARI, EPCARD}{6}\\
\ganttmilestone[]{DGA Courses}{6}
\ganttbar[bar height=.4]{Emulation Platform for Seq. ckt}{12}{35} \\
%ganttmilestone[]{Radiation Experiment}{12} \\
\ganttbar[bar height=.4]{Modelling and Analysis}{25}{42} \\
\ganttmilestone[]{Radiation Experiment}{20}
\ganttmilestone[]{Conference Paper}{30}
\ganttmilestone[]{Conference Paper}{39}\\
\ganttmilestone[]{Conference Paper}{33}\\
%\ganttbar[bar height=.4]{Novel Timing analysis techniques}{22}{38} \\
%\ganttmilestone[]{DAC}{36}\\
%\ganttbar[bar height=.4]{FPGA Implementation}{27}{38} \\
%\ganttmilestone[]{TRETS}{40}\\
\ganttbar[bar height=.4]{Tech. Demo.}{35}{42} \\
\ganttmilestone[]{Journal Paper}{42}\\
\ganttbar[bar height=.4]{Thesis Writing}{35}{46}
%\ganttmilestone[]{Thesis}{44}\\
%\ganttbar[bar height=.4]{The \emph{PolyOrbite} Project}{1}{44} 
%\ganttlink{elem0}{elem1}
%\ganttlink{elem0}{elem2}
%\ganttlink{elem0}{elem3}
%\ganttlink{elem3}{elem4}
%\ganttlink{elem5}{elem6}
%\ganttlink{elem5}{elem7}
%\ganttlink{elem7}{elem8}
\end{ganttchart}
\end{tikzpicture}
}

\caption{Timetable.}
\label{timetable}
\end{figure}

\end{center}

The work for the first conference paper is 75\% ready in which, we will present the new fault emulation strategy based on bits sensitivity, and it's feature and provide the results of the signature analysis.

In the second paper, we will present the high-level fault models for adder and multiplier~\footnote{Mr. Claude Thibeault is the first author}. 

We target one more conference paper. In this paper, we will provide the fault modeling with HMMs. We will give the details for signature generation for sequential circuits based on the new emulation technique, we will cover the  fault models and basic HMMs.



The journal paper is the ultimate target we would like to present the methodology and algorithms.  We would like to combine all the work in this article. Including, emulation technique, low-level signature generation, a comparison between the signatures from the radiation experiment and from the emulation setup, high-level modeling, create faulty FSMs and circuits, the utility of  VHDL entity, generation of the faulty components library, and resource utilization for the faulty components.  


