
%\section{High-level Fault Models Pseudo Codes}
\label{appendix}
\begin{itemize}

\item \textbf{Lost signal or event fault:} This kind of fault model we might observe when a certain process of a system stop working. For example, a 3-bit counter, the result of first two bits are accurate but the third flip-flop didn't respond.


\begin{lstlisting}[frame=single]  % Start your code-block

process (clk, rst)
 count <= "000";
 count <= count + 1;
end process;
\end{lstlisting}

\textbf{High-Level Model}

\begin{lstlisting}[frame=single]  % Start your code-block

process (clk, rst)
 count <= "00";
 count <= count + 1;
end process;
\end{lstlisting}



\item \textbf{Stuck Signal:} In this kind of fault model the signal is stuck to some fixed value. For example, the input of a FIR filter stuck to some unknown fixed value. 

 

\item \textbf{Variable change fault:} This fault behavior shows the situation in which either input, output, or intermediate calculation value changes to some other value. Now, consider a case as shown in Table~\ref{FMFIRA}, the expected faulty output of the FIR filter: $\{1, 6, 5, 16, 15\}$ due to variable change fault the faulty output observed: $\{1, 80, 5, 16, 15\}$. The signature observed from the variable change fault is only one position different from the expected signature (stuck-at-fault).  It means we cannot use the stuck-at-fault model to imitate this kind of faulty behavior. This is  the limitation of the stuck-at-fault (applies to other models as well). In this thesis, will make different fault-model to represent that signature.


\begin{table}[tb!]
\center
\caption{Variable Change Fault Models FIR}

\label{FMFIRA}
\scalebox{0.75}{
\begin{tabular}{|c|c|c|c|} 
 \hline
 \rowcolor{lightgray}
\shortstack {Faulty Value \\(Stuck-at-fault)} & \shortstack {Faulty Value \\ (Variable change)} & \shortstack {Signature \\ (Stuck-at-fault)} & \shortstack {Signature \\ (Variable Change)}   \\ 
\hline

 
 
 1& 1 & 0 & 0  \\
 \hline
 6 & 80 & 1 & -75 \\ 
 \hline
 
 5 & 5 & 0 & 0 \\
 \hline
 6& 16& 1& 1 \\
 \hline
 15 & 15  &  0& 0 \\

 \hline
 
 
\end{tabular}
}
\end{table}

\item \textbf{Changing the specified range of output:} This kind of faulty behavior, shows if we get the output which is out of the range of the expected value. For example, the 3-bit counter gives the output that is out of range of 3-bit counter, e,g., $out$ $<=$ $10$.

\item \textbf{Delayed Fault:} This fault explains the situation in which the output is delayed according to the pre-determined outputs. For example, the output of the counter comes after ten clock cycle delay.



\item \textbf{Invert Fault:} When the fault occurs, and it inverts the actual value to the faulty value, e.g., true to false and false to true. This kind of fault observes for \textit{if} and \textit{else}
conditional statements. In the following example, the \textit{if} statement start working on condition: \textit{false} instead of \textit{true}.


\begin{lstlisting}[frame=single]  % Start your code-block

if (A = B) then 
 out <= in1;
else
 out <= in2
end if;
\end{lstlisting}

\textbf{High-Level Model}

\begin{lstlisting}[frame=single]  % Start your code-block

if (false) then 
 out <= in1;
else
 out <= in2
end if;
\end{lstlisting}


\item \textbf{Short Circuit Fault:} The fault model makes the two independent variables dependent on each other. In the following example, \textit{out1 <= in1} and \textit{out2 <= in2}, but this fault makes the \textit{out2} dependent on the \textit{out1} result.


\begin{lstlisting}[frame=single]  % Start your code-block

if (A = B) then 
 out1  <= in1;
 out2 <= in2;
else
 out1 <= in3;
 out2 <=in4;
end if;
\end{lstlisting}


\textbf{High-level Model}
\begin{lstlisting}[frame=single]  % Start your code-block

if (A = B) then 
 out1 <= in1;
 out2 <=in1;
else
 out1 <= in3;
 out2 <=in3;
end if;
\end{lstlisting}




\item \textbf{Open Circuit Fault:} The signal can take any non-deterministic value.



\begin{lstlisting}[frame=single]  % Start your code-block

if (A = B) then 
 out1  <= in1;
else
 out1 <= in2;
end if;
\end{lstlisting}


\textbf{High-level Model}
\begin{lstlisting}[frame=single]  % Start your code-block

if (A = B) then 
 out1 <= A;
else
 out1 <= A;
end if;
\end{lstlisting}


\item \textbf{Stuck-Then:} This fault model describes the failure of the \textit{if-then-else} statement in which the \textit{else}
 statement never executed. In the following example, the \textit{else }
statement never execute.

\begin{lstlisting}[frame=single]  % Start your code-block

if (A = B) then 
 out1  <= in1;
else
 out1 <= in2;
end if;
\end{lstlisting}


\textbf{High-level Model}
\begin{lstlisting}[frame=single]  % Start your code-block

if (true) then 
 out1 <= in1;
else
 out1 <= in2;
end if;
\end{lstlisting}



\item \textbf{Stuck-Else:} This fault model represents the behavior in which the \textit{then} statement never execute.



\begin{lstlisting}[frame=single]  % Start your code-block

if (A = B) then 
 out1  <= in1;
else
 out1 <= in2;
end if;
\end{lstlisting}


\textbf{High-level Model}
\begin{lstlisting}[frame=single]  % Start your code-block

if (false) then 
 out1 <= in1;
else
 out1 <= in2;
end if;
\end{lstlisting}



\item \textbf{Dead Process:} This fault model describes the failure of the statement inside the \textit{process} statement. We can make the high-level model of this fault by changing the sensitivity list of the process statement.


\begin{lstlisting}[frame=single]  % Start your code-block

process(clk, rst)
 begin
 state 1 <= in1;
 state 2 <= in2;
 end process;
\end{lstlisting}


\textbf{High-level Model}
\begin{lstlisting}[frame=single]  % Start your code-block

process(constant='1')
 begin
 state 1 <= in1;
 state 2 <= in2;
 end process;
\end{lstlisting}



\item \textbf{Micro-operation fault}: This fault model represents the failure of the operation of the logical operator, arithmetic operator, and relational operator, e.g., xor, <=.   


\begin{lstlisting}[frame=single]  % Start your code-block

if (A = B) then 
 out1  <= in1 and in2 ;
else
 out1 <= in2;
end if;
\end{lstlisting}


\textbf{High-level Model}
\begin{lstlisting}[frame=single]  % Start your code-block

if (A = B) then 
 out1 <= in1 or in2;
else
 out1 <= in2;
end if;
\end{lstlisting}

\item \textbf{Swap Value Fault:} The fault model depicts  swapping the values between two variables.



\begin{lstlisting}[frame=single]  % Start your code-block
if (A = B) then 
 out1  <= in1;
 out2 <= in2;
else
 out3 <= in3;
 out4 <= in4;
end if;
\end{lstlisting}


\textbf{High-level Model}
\begin{lstlisting}[frame=single]  % Start your code-block

if (A = B) then 
 out1  <= in2;
 out2 <= in1;
else
 out3 <= in4;
 out4 <= in3;
end if;
\end{lstlisting}

\end{itemize}

