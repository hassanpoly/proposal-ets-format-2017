% Résumé du mémoire.
%
%   Le résumé est un bref exposé du sujet traité, des objectifs visés,
% des hypothèses émises, des méthodes expérimentales utilisées et de
% l'analyse des résultats obtenus. On y présente également les
% principales conclusions de la recherche ainsi que ses applications
% éventuelles. En général, un résumé ne dépasse pas quatre pages.
%
%   Le résumé doit donner une idée exacte du contenu du mémoire. Ce ne
% peut pas être une simple énumération des parties du mémoire, car il
% doit faire ressortir l'originalité de la recherche, son aspect
% créatif et sa contribution au développement de la technologie ou à
% l'avancement des connaissances en génie et en sciences appliquées.
% Un résumé ne doit jamais comporter de références ou de figures.
\chapter*{RÉSUMÉ}\thispagestyle{headings}
\addcontentsline{toc}{compteur}{R\'ESUMÉ} \selectlanguage{french}
%Les systèmes informatiques de nos jours  se caracterisent de plus en plus par une plus grand complexite, ce qui demande des méthodes nouvelles et plus efficaces au niveau de l'automatisation  de la conception.  
%L'espace, plus particulièrement, représente un environnement délicat: sans une atmosphère à l'abri de rayonnements ionisants ou de radiation particulaire, l'électronique actuel à base de CMOS est soumis à des défauts transitoires, à une diminution de la performance global, à une usure précoce et, finalement, à la défaillance irrécupérable du système. 
%%Les approches traditionnelles adoptées pour garantir la fiabilité et la durée de vie prolongée sont basées sur la redondance modulaire triple. 
%%Cependant, ces solutions sont coûteuses en termes de ressources et nécessitent un compromis prudent, car ils augmentent la complexité et l'étendue du système, l’exposant à un risque élevé de surchauffe et de radiation. 
%En outre, les systèmes qui sont indispensables, de durée limitée et où l'accès est limité, doivent être en mesure de faire face à des situations où  il ne peut y avoir recourt  à l'intervention humaine.
%Par conséquent, il y a un intérêt naissant autour des systèmes informatiques dont la capacité d'adaptation seraient particulièrement ajusté aux  appareils de haute performance comme ceux  employés dans l'aérospatiale. 
%Les systèmes informatiques auto-adaptatifs procurent un potentiel inégalé et de grandes promesses au niveau de la création d'une nouvelle génération d’ordinateurs intelligents et plus fiables, ce qui permet à les systèmes informatiques modernes et futurs  de pouvoir répondre à des objectifs contradictoires.
% Puisant dans les domaines de l'intelligence artificielle, de l'informatique autonome et des systèmes reconfigurables, nous visons à développer des systèmes informatiques auto-adaptatifs pour l'aéronautique. 
% Nous allons commencer par créer un cadre d'optimisation, permettant un compromis autonome et adaptif entre la fiabilité et la performance. 
% Au point de vue du cadre d'optimisation, nous allons nous appuyer sur l'exploitation de modèles graphiques probabilistes et sur la théorie de la décision. 
%% Nous allons également examiner les multiples possibilités entourant les (formules) d'apprentissages. 
% En raison de contraintes strictes de temps imposées par les systèmes aérospatiaux, nous allons concevoir notre cadre d'optimisation sur la base d'un système de criticité mixte, dans laquelle les décisions d'adaptation et d’optimisation sont faites en temps réel et de façon fiable. 
% Par conséquent, notre méthodologie fournira des systèmes informatiques dont la capacité sera de répondre aux contraintes de performances et de temps,  permettant la détection et la résolution des fautes de façon simultanée. 
% Nous allons d'abord mettre en place un prototype FPGA dans notre laboratoire pour vérifier la performance de notre système. 
%Nous prévoyons tester le prototype sur le terrain à l’aide de ballons scientifiques de haute altitude et/ou sur une plate-forme de CubeSat. 
% Notre objectif est d'améliorer l'efficacité, la tolérance aux fautes, et la capacité de calcul des systèmes informatiques aérospatiaux. 
% Le but de cette recherche est de  permettre la création d'une nouvelle génération de systèmes capables d'exécuter leur taches de manière autonome pour des périodes de temps prolongés, afin de favoriser  les experiences scientifiques dans l'espace à moindre coût et de façon simplifié.


