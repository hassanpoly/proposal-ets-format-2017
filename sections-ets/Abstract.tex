



\begin{abstract}{FPGA, SEU, Modeling}
Cosmic radiation   (CR)   produces soft errors on aircraft’s   embedded electronic system. Aircraft flying   at altitude of  55,000   feet,   and cross polar   routes   (North or   South latitudes)   are prone to neutron flux. CMOS ­based  
electronics are   subject to   hard and soft errors. Equipment protection   against   CR   is becoming critical. The solutions to   protect electrical systems  
from cosmic rays are   developed but
Unfortunately,   these solutions   are costly,   time,   and energy   consuming, e.g.,   dedicated heavy   conductive electrical   path. To progress   towards the   safe and   efficient aircraft,   it is   now necessary to   anticipate the   aircraft embedded systems constraints in   the early   phase of   the aircraft   design. 

In this project,   we will   study the   novel algorithms   and methodology   for high   levels modeling   of cosmic  
radiation impacts on   the aircraft. This thesis aims   at defining a   novel approach for   high levels   modeling of   cosmic radiation   impacts on   digital sequential circuits.   In   particular,  our   thesis   intends   to   provide   solutions   able   to   mitigate   multiple   problems,  
including  a)  Bits relative sensitivity based emulation   strategy   for   complex   systems, b) Signature generation on the FPGA-based emulation platform, c) analyze the signatures from the radiation-based experiment, d) Modelling the faulty behavior of the sequential circuit observed at lower level of abstraction to abstract it at the higher level of abstraction. The methodology adopted in this thesis is based on the Hidden Markov Model (HMM). HMM is used for modeling and analysis of the sequential circuits. Faulty behavior of the circuit is modeled with the HMM. HMM helped to identified the fault model.



\end{abstract}