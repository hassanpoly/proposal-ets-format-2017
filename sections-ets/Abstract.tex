

\begin{abstract}{FPGA, SEU, Modeling.}
Cosmic radiations   (CR)   produce soft errors in aircraft's embedded electronic systems. Aircraft flying at an altitude of  55,000   feet,   and cross-polar routes   (North or   South latitudes)   are prone to neutron flux. CMOS based electronics are subject to hard and soft errors. Equipment protection against   CR   is becoming critical. The solutions to   protect electrical systems  
from cosmic rays are   developed but
unfortunately,   these solutions are costly,   time,   and energy consuming, e.g.,   dedicated heavy conductive electrical path. To progress towards the safe and efficient aircraft,   it is now necessary to anticipate the aircraft's embedded system constraints in the early phase of the aircraft design. 
In this project,   we will   study the   novel algorithms   and methodology   for high-level modeling   of cosmic  
radiation impacts on   the aircraft. This thesis aims at defining a   novel approach for high-level modeling of digital sequential circuits.   In   particular,  our   thesis   intends   to   provide   solutions   able   to   mitigate   multiple   problems, i.e., a) Signature generations on the FPGA-based emulation platform, b) Analyze the signatures from the radiation-based experiment, c) Modeling the faulty behavior of the sequential circuit from the signatures observe at lower abstraction level to abstract it at the higher level of abstraction, and d)  Improve bits relative sensitivity difference based emulation strategy for FPGA based emulation platform. The methodology adopted in this thesis for modeling and analysis of the sequential circuits is based on the Hidden Markov Model (HMM). 
\end{abstract}