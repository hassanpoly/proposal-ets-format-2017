%%%%%%%%%%%%%%%%%%%%%%%%%%%%%%%%%%%%%%%%%%%%%%%%%%%
% DOCUMENT CLASS DECLARATION
%%%%%%%%%%%%%%%%%%%%%%%%%%%%%%%%%%%%%%%%%%%%%%%%%%%
%% Use the following options:
% \documentclass[paper type% ("letterpaper" required)
% , one or two sided% ("oneside" or "twoside")%
% , font size% ("12pt" required)%
% , document type% ("these", "memoire", "memoireprojet" or "thesepararticles")%
% , document language ("francais" or "english)%
% , addition options% ("creativecommons" if the document is under the creative commons license, "hyperref", "withAlgo2e" to use algorithm2e package with proper formating)
%]{thETS}




%% Exemple with a Ph.D thesis under creative commons, using hyperref
\documentclass[letterpaper%
, twoside%
, 12pt%
,these%
, english%
,creativecommons,hyperref%
]{thETS}

\usepackage{siunitx}
\usepackage{tikz}
\usepackage{pgf}
\usepackage{mdwlist}
\usepackage{multirow}
\usepackage{booktabs}

%%%%%%%%%%%%%%%%%%%%%%%%%%%%%%%%%%%%%%%%%%%%%%%%%%%
% IMPORTANT: PRINTING WITH THE PROPER MARGINS
%%%%%%%%%%%%%%%%%%%%%%%%%%%%%%%%%%%%%%%%%%%%%%%%%%%
%% If you create a pdf with pdftex, and print it using acrobat reader, set the
%% "scaling" option to "none" to print with the proper margins.
%%%%%%%%%%%%%%%%%%%%%%%%%%%%%%%%%%%%%%%%%%%%%%%%%%%


%%%%%%%%%%%%%%%%%%%%%%%%%%%%%%%%%%%%%%%%%%%%%%%%%%%
% DECLARATION OF AN ADDITION LIST OF REFERENCES
%%%%%%%%%%%%%%%%%%%%%%%%%%%%%%%%%%%%%%%%%%%%%%%%%%%
%% Exemple of an additional list of references called "refs"
% "refs" is used as a suffix to all bibliography related commands
\newcites{refs}{LIST OF REFERENCES}

%%%%%%%%%%%%%%%%%%%%%%%%%%%%%%%%%%%%%%%%%%%%%%%%%%%
% TITLE PAGE OPTIONS
%%%%%%%%%%%%%%%%%%%%%%%%%%%%%%%%%%%%%%%%%%%%%%%%%%%

\title{METHODOLOGY AND ALGORITHMS FOR HIGH-LEVEL MODELLING OF COSMIC
RADIATION IMPACTS ON ELECTRICAL SYSTEMS}

\author{HASSAN ANWAR}
\authorcopyright{HASSAN ANWAR}

\datesoutenance{"20-December-2017"}

\datedepot{"02-December-2017"}

\directeur{M. }{Claude Thiebeault}{Départment de génie électrique  à l’École de technologie supérieure}

%\directeur{Mrs.}{Prénom Nom}{Nom du département et institution}

\codirecteur{Mrs.}{Yvon Savaria}{Départment de génie électrique  à l’École Polytechnique de Montréal}

%\codirecteurB{M.}{Prénom Nom}{département et institution}

\president{M.}{First Name Last Name}{Department and institution}

\examinexterne{M.}{First Name Last Name}{Department and institution}

%\jury{Mme.}{Prénom Nom}{département et institution}{}


%%%%%%%%%%%%%%%%%%%%%%%%%%%%%%%%%%%%%%%%%%%%%%%%%%%
% CHANGING THE NAME OF THE DIPLOMA
%%%%%%%%%%%%%%%%%%%%%%%%%%%%%%%%%%%%%%%%%%%%%%%%%%%
%% It is possible to change the name of the diploma by redefining
% the command \lediplome, as follows:
%\renewcommand{\lediplome}{OF A MASTER’S DEGREE\\WITH THESIS IN ELECTRICAL ENGINEERING\\M.A.Sc.}

\listfiles

%%%%%%%%%%%%%%%%%%%%%%%%%%%%%%%%%%%%%%%%%%%%%%%%%%%
% ACTUAL DOCUMENT
%%%%%%%%%%%%%%%%%%%%%%%%%%%%%%%%%%%%%%%%%%%%%%%%%%%
\begin{document}

\pagenumbering{Roman}

%%- Title page -%%
\maketitle

%%- Jury presentation -%%
\presentjury

%%- Foreword -%%
%\begin{foreword}
%
%\lipsum[1] % Texte de remplissage pour donner un exemple de la mise en page
%
%\end{foreword}



%%- Acknowledgements -%%
%\begin{acknowledgements}
%
%\lipsum[1] % Text filling, to have an example of the layout
%
%
%\end{acknowledgements}


%%- Summary -%%

%\begin{summary}{French title}{mot-clé1, mot-clé2}
%
%\lipsum[1] % Text filling, to have an example of the layout
%
%\end{summary}


%%- Abstract -%%





\begin{abstract}{FPGA, SEU, Modeling.}
Cosmic radiations   (CR)   produce soft errors in aircraft's embedded electronic systems. Aircraft flying at an altitude of  55,000   feet,   and cross-polar routes   (North or   South latitudes)   are prone to neutron flux. CMOS based electronics are subject to hard and soft errors. Equipment protection against   CR   is becoming critical. The solutions to   protect electrical systems  
from cosmic rays are   developed but
unfortunately,   these solutions are costly,   time,   and energy consuming, e.g.,   dedicated heavy conductive electrical path. To progress towards the safe and efficient aircraft,   it is now necessary to anticipate the aircraft's embedded system constraints in the early phase of the aircraft design. 
In this project,   we will   study the   novel algorithms   and methodology   for high-level modeling   of cosmic  
radiation impacts on   the aircraft. This thesis aims at defining a   novel approach for high-level modeling of digital sequential circuits.   In   particular,  our   thesis   intends   to   provide   solutions   able   to   mitigate   multiple   problems, i.e., a) Signature generations on the FPGA-based emulation platform, b) Analyze the signatures from the radiation-based experiment, c) Modeling the faulty behavior of the sequential circuit from the signatures observe at lower abstraction level to abstract it at the higher level of abstraction, and d)  Improve bits relative sensitivity difference based emulation strategy for FPGA based emulation platform. The methodology adopted in this thesis for modeling and analysis of the sequential circuits is based on the Hidden Markov Model (HMM). 
\end{abstract}

%%- Table of contents -%%
\tableofcontents
 
%%- List of tables -%%
\listoftables


%%- List of figures -%%
\listoffigures


%%- List of abbreviations -%%
\begin{listofabbr}[3cm]
\item [DUT]  Design Under Test 

\item [FPGA]  Field-Programmable Gate Array 
HDL Hardware Description Language 


\item [SBU] Single-Bit Upset 
\item [SEE] Single-Event Upset 
\item [SEL] Single-Event Latch-up 
\item [SER] Soft Error Rate 
\item [SET] Single-Event Transient 
\item [SEU] Single-Event Upset 
\item [SRAM] Static Random Access Memory 
\item [TMR] Triple Modular Redundancy 
\end{listofabbr}


%%%- List of symbols -%%
%\begin{listofsymbols}[3cm]
%\item [a] Première lettre de l'alphabet
%\item [A] Première lettre de l'alphabet en majuscule
%\end{listofsymbols}


\cleardoublepage

\pagenumbering{arabic}

% Marginpar to the left of the document
\reversemarginpar

%%%%%%%%%%%%%%%%%%%%%%%%%%%%%%%%%%%%%%%%%%%%%%%%%%%
% THESIS EXAMPLE
%%%%%%%%%%%%%%%%%%%%%%%%%%%%%%%%%%%%%%%%%%%%%%%%%%%

\chapter{Introduction}

%Prepare the tests on the EPICEA active sub-system (to be subcontracted)
%- Develop the active EPICEA sub-system for a SRAM-based CR emulation strategy for complex systems. Use
%of FPGA circuits from Xilinx allowing CR emulation (without CR illumination) by bit flipping for preparing
%the tests to be performed by the CR test subcontractor

\section{Context}
\label{introduction}

This thesis is part of a wider project, called EPICEA (Electromagnetic Platform for lightweight 
Integration/Installation of electrical systems in Composite Electrical Aircraft), in collaboration with several 
companies and agencies (Bombardier, Isoneo, ARTTIC, ONERA, AXESSIM, FOKKER ELMO BV, IDS). The EPICEA project will approach
numerous avionic engineering design issues in the advancement of future aircraft, aiming at a significant
reduction of energy consumption through more electrical aircraft and systems integration. The EPICEA project strives to understand the electromagnetic (EM) issues on composite electric aircraft (CEA). This includes the analysis
and characterization of EM coupling, interconnects, and cosmic radiations (CR) on electrical systems together
with new concepts of antennas design to maintain performance in a composite environment without modifying
aircraft aerodynamics.

This thesis contribution to the EPICEA project is "the study of CR effects on aircraft electrical systems based on the reconfigurable fabric, e.g., Field Programmable Field array (FPGA)." This work focuses on the extraction of the faulty response (signature) from the sequential digital circuit implemented on an FPGA-based platform. This extraction will be performed to help investigate the effects of CR on an embedded electronic system of the aircraft and to model and analyze the faulty behavior of the sequential digital circuits. This thesis aims at developing the high-level fault model that will use to investigate the effects of radiation on the aircraft flying at the altitude of 50,000 feet or above. 

\section{Problem Statement}

A soft error will occur when a radiation event causes enough charge disturbance to flip the state of a logic gate, memory element, or flip-flop. The soft errors are also referred as radiation-induced faults, e.g., neutrons from the cosmic rays. The fault caused by the SEUs have implications on the behavior of the system. However, the soft errors do not break the silicon but can corrupt the underlying functionality and produce severe consequences if they are successfully propagated to the primary outputs or stored by the memory cell. In this work, we will evaluate the faulty behaviors of the sequential circuit, given the faulty sequential circuit behavior, and model it to the high-level of abstraction. We want to make the high-level fault model to see the severity of the faulty behavior induced by the SEUs.



This thesis helps to find; at high altitude when aircraft gets more exposure to the radiation the reliability of the FPGA based embedded systems is a primary concern; the phenomena causing faults in the FPGA based embedded systems should be studied in a way to know early in the embedded electronic design of the aircraft if mitigation strategies are required to deal this higher radiation level. 

In order to accomplish our tasks, we need to understand the space radiation environment which has two preliminary sources --- galactic cosmic radiation and solar energetic particles~\cite{SWE20216}. The galactic cosmic radiation from outside the solar system consists mostly of energetic protons and heavy ions, e.g., iron. Solar energetic particles are commonly associated with the solar flare events and primarily dominated by the proton.  The space radiation is an unavoidable space weather phenomena. This research thesis is solely concern with the aircraft flying at the altitude of 50,000 feet or above, where the vulnerability of the circuit is due to the neutrons~\cite{xilinnseu}. The cosmic rays are originating in outer space and travel at nearly the speed of light and strike the earth from all directions. These cosmic radiations are ranging from lightest to heaviest elements in the periodic table. When these high-energy cosmic rays interact with the earth's magnetosphere, neutrons are generated, often referred to as an air shower~\cite{lesea2005rosetta}.  An intense neutron environment exists at higher altitudes in the atmosphere, 10 km to 40 km.  Long-haul aircraft are flying at the altitudes of 50,000 feet nearly 15 km at the latitude of \ang{60} under the influence of greatest neutron flux approximately 500 times that a ground-based observer in Newyork City~\cite{lesea2005rosetta}. When these neutrons interact with the semiconductor, e.g., silicon they can produce secondary particles if these secondary particles are charged and they can generate the trails of an electron-hole pair of a few microns in length and if this trail happens near the PN-junction in a transistor, a voltage spike can generate. This voltage spike is enough to change the state of a memory cell or flip-flop, results in a drastic consequence and cause Single Event Upsets (SEUs) also called soft-error in SRAM-based FPGAs. The impact and implications of the soft-errors due to the interaction of neutrons with the avionics embedded system are now recognized as an area of active research. Especially, the incident happened with the Qantas Flight Airbus A330-303 flying from Singapore to Perth went under the two terrifying dives due to the malfunction of the on-flight computer. After, the investigation it revealed that high-energy particles from the outer space --- were the responsible for the malfunction of the computer. And, the potential triggering event was the single-event effect (SEE) interacting with one of the integrated circuits (ICs) within the CPU module. 




%
%The Neutrons are generated in the atmosphere when cosmic rays in the outer space, strike Earth from all the direction. When these neutrons interact with the semiconductor e.g., silicon they can generate secondary particles, if these secondary particles are charged and they are able to generate the trails of electron-hole pair of a few microns in length and if this trail happens near the PN-junction in a transistor, a voltage spike can generate. This voltage spike is enough to change the state of a memory or flip-flop.

Therefore, fault management strategies are essential to apply on the aircraft's embedded systems. Before need to know to apply the mitigation techniques early in the embedded electronic design, if we know the high-level fault model of the FPGA-based circuits at high-level of abstraction that facilitate without going into the detailed simulation get the faulty behavior of the component at high-level.

  




%\begin{figure}
% \centering
%  \captionsetup{justification=centering}    
%   \includegraphics[scale=0.4]{figures/img/neutron-flux.png}
%   \caption{Neutron Flux at 40,000 Feet.}
%\label{fig:neu-flux}
%\end{figure}





\section{Research Objectives}


The main objective this thesis --- is to develop a methodology for modeling the faulty sequential circuits to model the output of the soft error problem at a behavioral level. We plan to use the high-level modeling techniques, e.g., Hidden Markov Model, to model the faulty behavior of the sequential circuits. The faulty behavior model will be based on analytical expressions, the signature (circuit fault response),derived from  experimental data generate by performing fault emulation on the FPGA. In this work, a novel approach will be introduced to analyze the fault origination and propagation in the case of faulty sequential circuits, and will allow developing new models that can accurately be used to estimate the severity of the faulty behavior from the signatures of the faulty circuits. The objectives of the present Ph.D. thesis are explained below: 

\begin{itemize}


\item{The objective of this thesis is to develop a faulty behavior model
at a high-level of abstraction. 


\item The objective of the thesis is to construct a model from the faulty response observed at low-level circuit fault emulation.

\item{Develop a library of the faulty behavior model of the sequential circuit
components comprising a Simulink model. So, each time designers need to analyze potential faulty behavior of a circuit at a high-level of abstraction, designer can utilize the faulty components from the library}.

\item The sub-objective of developing a behavioral fault model of a circuit is to generate a library of faulty components reusable at high-level of abstraction}.



\item{The model has the capability (a feature) to produce the faulty output of a circuit, and the probability of occurrence of faults.}

\item The purpose of this model is to find the severity of a faulty behavior of the sequential circuit. 

\item The model is used for the signature analysis for the sequential circuit.






%
%\item{The developed models could be used to
%replace any component of the entire circuit with faulty versions of the components described
%at a high-level of abstraction. The purpose of doing so to ensure that the effect of faulty behavior of each
%component on a system could be analyzed at a high-level of abstraction and the mitigation
%technique could be used to improve the robustness of critical parts of the design.}
%
%
%\item{The model will have the capability to compute the signature, also provide the worst-case input test vector, which has the highest probability to generate the faulty output, for any given sequential circuit. The model for sequential logic will have the ability to measure the maximum and average signature probabilities to construct a novel probabilistic model.}
%
%\item{Develop an abstraction of hardware signature that can be integrated into the high-level model, e.g., Simulink models.}
%
%\item{The goal of this research is to develop an approach for modeling the faulty behavior of a
%digital sequential circuit in the presence of the fault injection. The concept behind the fault injection process is to accelerate the occurrences of the signatures in the system to evaluate its functional behavior under the influence of expected faults on the
%FPGA-based systems.}





\end{itemize}




\subsection{Challenges}
Inorder to to achieve the above mentioned objectives. The main challenges we foresee are:
\begin{itemize}

\item Make a model at higher-level of abstraction from the data extracted at a lower level that represent the behavioral model of the respective signatures.
\item Develop a flow to convert faulty behavioral response of a sequential circuit into respective high-level model. This flow will be extensively validated by the fault emulation on FPGA. Develop a  fault behavioral model of different sequential circuits, e.g., counter, and FIR filter.

\item Develop a relationship between the bit-flip and the fault-model.

\end{itemize}


  

\section{Contribution}


This research thesis proposes a fault behavior model with the modeling techniques, e.g., Markovian-analysis in a novel way (utilize hidden Markov model (HMM) to represent faulty behavior of the sequential circuits). The Markovian system analysis will use to synthesize the faulty output of a circuit at high-level of abstraction. The existing faulty behavior model of the sequential circuits are not accurate as mostly models were based on simulation and on fault occurrence assumption. 

The previous work on soft error analysis and evaluation were focused on the circuit level. Most of the work done based on the interaction of the cosmic rays with the silicon atoms and its analysis, prediction of error, estimating the model of transistors and circuit layout structure that can tolerate radiations effects. On circuit level researchers were more focused on the generation and propagation of the radiation-induced transient pulses. The effects of these pulses are simulated with the help of SPICE. These fundamental analyses are essential and provide insight of soft error rate. However, they can not take the overall erroneous behavior of the circuit at the architecture level --- the errors produced at the low level propagate to the high level of abstraction. Further, the increased complexity of integrated circuits, and integration of several components, it is more difficult now to do the low-level circuit analysis and make the model at low-level. It is a more viable solution to find the faulty response of the circuit at the low level and model it to the high-level of abstraction. The high-level design suppresses a lot of unnecessary details, reduce the complexity, and its closer to the designer way of thinking. Our problem is unique in terms that it involve emulation at the circuit level and behavioral modeling as well, as can be depicted from the Figure~\ref{fig:ychart}.


\begin{figure}[tb!]

 \centering
  \captionsetup{justification=centering}    
   \includegraphics[scale=0.8]{Figures/ychart-block.pdf}
   \caption{Contribution of this work on Gajski-Kuhn chart.}
\label{fig:ychart}
\end{figure}



The objective of the thesis is to construct a high-level model from the faulty response observed at low-level circuit. This model provides the response of the system under fault. It can be described with Y-chart as shown in Figure~\ref{fig:ychart}, in which the high-level analysis (Behavioral View) based on the faulty response at the circuit level. At high-abstraction level the detailed of the implemented design is hidden from the designer making it more algorithmic and model based problem. In this work, we propose a solution --- simulate the low-level faulty response to the high-level behavioral model. The solution is based on the Hidden Markov Model, that uses the concept of hidden states  (stuck-at-fault) and observed states (signature) to find not only the hidden states of the faulty system but also accurately model the observed states. 


\begin{itemize}

%\item In this work, we will construct a high-level fault model that model the signature of the sequential circuits (ITC'99 benchmark). The high-level behavioral model based on the signatures from the sequential circuits.

\item This work will focus on the soft-error susceptibility for sequential circuits which are different from the combinational circuits. The error in the sequential circuit can be propagated back to the inputs, or circuits outputs can be affected for several consecutive clock cycles making the design more vulnerable.  
\item Our model will have the capability can efficiently model the worst case signature and the input vector correlated with this signature, through probabilistic modeling (HMM) and have the ability to measure the maximum and average signature probabilities. We also try to establish a relationship between the FPGA bits emulation information to the faulty models.



\end{itemize}

%%% Local Variables:
%% mode: latex
%% TeX-master: "../Document"
%% End:





%%- Uncomment the literature review for a thesis by articles -%%
%\begin{literaturereview}

%\end{literaturereview}

%%- First demo chapter -%%
\chapter{Related Work and Prior Work}


This chapter is dedicated to the fundamental concepts and current research in different areas related to this project: radiation environment, radiation effects on SRAM-FPGAs, single-event upsets, fault-injection, signatures,  and behavioral fault modeling. All of these topics are relevant for the purpose of this research --- ideally, places itself as an attractive research project.





%The potential analysis of a malfunction due to radiation encourages manufacturers to invest more and more in the development of the design tools that allows the choice of the mitigation strategies. In the 1980s,
%components have begun to be developed for space applications and Aviation
%In the 1990s, the explosion of the development of consumer electronics multiplied
%Commercial Off-The-Shelf (COTS) components, which are
%economical and available in large quantity. Further,  the reduction
%of the transistor sizes, the intrinsically small transistor is less sensitive to
%radiation. But because of its smaller sensitive surface, the opposite effect is observed at the level of an integrated circuit. Indeed, the sensitivity of the components increases while the ability of
%engraving decreases despite the gain brought by the miniaturization. The two main factors
%of this conjecture are the increase in the density of integration and the
%threshold voltages of the transistors. For the last ten years or so, studies have been able to provide the evidence of the potential sensitivity of integrated circuits at high altitude commercial flights \cite{normand1998extensions}. This is particularly the case for complex components with high integration such as
%those based on Static Random Access Memory (SRAM) \cite{baumann2005radiation}, which is the target technology for this project. The aerospace industry is looking for the solutions that offer an acceptable compromise between radiation reliability and the system cost, including those based on user-programmable integrated circuits based on static memory. The search for such solutions has implications for the complete flow of design and development of embedded systems. Indeed, this requires the designers to check design robustness to the radiations design. For this purpose, simulation tools that allow the injection of faults in the design is being used. But simulation-based verification has its limits, long testing time is added to the conventional verification, which is already considered to the bottleneck of the design process. The system, once implemented, should also be subject to costly laboratory testing using particle accelerators. To provide a cost-effective fault emulation techniques were developed by our research group ~\cite{hobeika2014multi,  bocquillon2009evaluation,souari2015optimization}.



\section{Radiation Environment}


The electronic systems operate in space are exposed to radiations. The first evident were observed in the sixties, but it was difficult to separate soft-error from the other form of interference. The first evidence of malfunctions in electronic circuits embarked in spacecraft caused by the radiations reported in May 1979 \citep{may1979alpha}. In the outer space, there are three main types of radiative sources that effect the Earth's atmosphere.

\begin{itemize}

\item Galactic cosmic rays.

\item Radiation from the sun, i.e., solar wind and solar flares.

\item Earth's magnetic field, e.g., magnetosphere and radiation belts.

\end{itemize}


\subsection{Galactic Cosmic Rays} 

The origin of cosmic rays is hardly known. However, we have information about they are energetic particle spread throughout the galaxy including the solar system \citep{SWE20216}. This radiation comes from sources present "in and out" of our
universe. Interactions of these radiations with an interstellar matter, shock waves, and electromagnetic fields, accelerate these radiations. As a result, at the scale of our
solar system, it appears as an isotropic angular distribution, and the particles are ionized~ \citep{SWE20216}. The sun is also the origin of some of these particles, but most of them come from the galactic sources known as galactic cosmic rays (GCR).
Cosmic radiation was discovered by V. Hess in 1912 through measurements made from balloon probes \citep{cronin1999cosmic}. The cosmic rays constitute of 85\% of the proton that is nuclei of the hydrogen atom, 12\% are alpha particles that are helium nuclei, and the others are electrons and nuclei of heavier atoms. The energy of cosmic rays ranges from 1GeV to 108 TeV. 


\subsection{Radiation from the Sun} 

The sun alone accounts for 99.8\% of the total mass of the solar system, the remaining 0.2\% other planets. The sun is mainly composed of hydrogen 90\% and helium 8\%~\citep{wikisun}. In the center of the sun, thermonuclear fusion reactions convert hydrogen into helium. The energy produced on this occasion is emitted in the form of radiation and particles. The radiation produced by the sun has two essential sources: solar flares and solar winds. 

\subsubsection{Solar Flares} 

The activity level of the sun is never constant but follows a cyclical variation that composed of active years followed by calm (no activity) years. The period of recent solar cycles has
varied between 9 and 13 years, with an average of about 11 years \citep{nasa}. The last maximum solar activity is observed in April 2014 period of the "solar-cycle" started in 2008 \citep{nasa}. The activity of solar cycles is frequently measured by the "number of sunspots observed." The first sunspot was observed in 1610 by Galileo \citep{nasa}. The number of sunspots varies cyclically, and most sunspots events are
generating protons appear during active solar years. During a solar cycle of 11 years, there are 4 years of low activity, and 7 years of strong activity, punctuated by sporadic emissions of large particle fluxes, In this radiative context, there are two types of solar flares: \\
- Proton solar flares, lasting from few hours to few days, and primary emission consists of protons of significant energy (up to
a few hundred MeV). \\
- Solar flares: the main emission consists of heavy ions.

\subsubsection{The Solar Wind}

The solar wind begins in the solar corona, where temperatures are in millions of degrees, gives electrons enough energy to allow them to
escape from the gravitational field of the Sun. In reaction, protons and heavy ions are also ejected. The Sun evacuates about $10^{14}$ kilograms of material each day during solar wind. 


\subsection{The Earth's Magnetic Field}

The Van Allen radiation belts are two toroidal zones of energetic particles that are held around the magnetic field of the Earth. The magnetosphere is the region surrounding a celestial object in which the
physical phenomena are dominated or organized by its magnetic field. These radiation belts are composed of energetic protons and electrons coming from solar wind and cosmic rays. There are two belts named --- inner belt and outer belt. The inner belt is composed of protons and electrons and outer belt is composed of energetic electrons~\citep{barth2003space}


\subsection{Influence of Radiations}
The previously mentioned sources of particulate matter influence only persons and equipment
outside the Earth's atmosphere. However, since 1992, the year of observation of the first
bit-flipped in-flight memory~\citep{taber1995investigation, taber1993single} observed. It has been demonstrated that the particles responsible for these
phenomena are atmospheric neutrons~\citep{leray2004atmospheric, jedec2006measurement}.
One more incident reported in~\cite{SWE20216} about the neutron influenced aircraft's operation of the flight from Singapore to Perth started incorrect values, after the investigation: it revealed that the SEE resulting from the high-energy atmospheric neutrons were interacting with the integrated circuits.
Atmospheric neutrons are the result of the collision of cosmic radiation,
mainly protons, with the atoms present in the upper atmosphere of the Earth. The result of these collisions is either the formation of ionized particles or a nuclear reaction which
produced mainly neutrons, protons, electrons, etc. This phenomena is called
"Atmospheric shower" as shown in Figure~\ref{shower}. The product resulting from these collisions is the generation of
protons, neutrons, muons, etc. Since neutrons are uncharged particles, they do not
cause  - operational failure directly of electronic devices. However, they can generate kernels of recoil in
the material they cross. When these ions originate in the active zones of the integrated circuit, they can modify the normal behavior of the components~\cite{normand1998extensions}.
%


\begin{figure}[h]
 \centering
  \captionsetup{justification=centering}    
   \includegraphics[scale=0.75]{Figures/showerplusaircraft.png}
  \caption{Cosmic rays shower~\cite{ziegler1996ibm}.}
\label{shower}
\end{figure}


\subsubsection{Neutron interactions}



The neutron is a particle without electric charge but with a mass. When a high-energy neutron moves through a silicon substrate, charged particles are produced. If the charge of these particles is sufficient enough, then they can alter the state of the static memory.  For avionics the \textbf{neutrons} are the dominant particles~\citep{xilinnseu}. Neutrons are measurable at the altitude of 330 KM and their density increases until it reaches around 40 km altitude. An intense environment exists at the altitude of 10 km to 40 km. The density decreases below the 40 km altitude and about 500 times lesser at ground level as compare to 40 km altitude. Until 90's the neutrons with energy above 100 MeV were considered dangerous for electronics components. But due to the shrinking the transistor size, circuits are now become more sensitive to the low energies as well.   

\subsection{Fault Caused by Cosmic Rays in Digital Circuits}


Ionizing particles can interact with an embedded electronics and have critical consequences for the success of a space mission or avionics altitudes flights. The interaction causes two types of effects ---
single event effects (SEE) (caused by a single particle) and the total ionizing dose (long-term radiation effects, mostly due to electrons and protons). In this research work, we mainly focused on the SEEs.


A Single Event Effect (SEE) results from a single energetic particle. When the particle strikes a sensitive node in a semi-conductor device, the ionization by the particle might produce a current pulse inside the device, which might cause soft or hard errors in the configuration memory of the device. Results in data corruption, transient disturbance, high current conditions (non-destructive and destructive
effects). If SEE cannot handle well cause unwanted functional interrupts or in worst case catastrophic failures. Commonly, SEEs include: single event upset (SEU), single event latch-up (SEL), single event burn-out (SEB), and single event transient (SET) etc as mentioned in Table~\ref{SEE-Summary}. 


\begin{table}
\caption{Single Event Effects Summary~\citep{manuzzato2010single}}
\centering
\label{SEE-Summary}
\scalebox{0.7}{

   \begin{tabular}{c|c|c}
         \toprule
    \hline
     
     Single Event Upset (SEU)                  & corruption of the information \\ & stored in a memory element            & Memories, latches in logic devices                                  \\ \hline
    
    Multiple Bit Upset (MBU)                  & several memory elements \\ & corrupted by a single strike                & Memories, latches in logic devices                                  \\ \hline
    Single Event Functional Interrupt (SEFI) & corruption of a data path      & Complex devices with built-in state       \\ \hline
    Single Hard Error (SHE)                   & unalterable change of state in\\ & a memory element                     & Memories, latches in logic devices                                 \\ \hline
    Single Event Transient (SET)              & Impulse response of certain\\ & amplitude and duration                  & Analog and Mixed Signal circuits                      \\ \hline
    Single Event Disturb (SED)                & Momentary corruption of the\\&information stored in a bit             & combinational logic, latches in logic devices                       \\ \hline
    Single Event Latchup (SEL)                & high-current conditions                                              & CMOS, BiCMOS devices                                                \\ \hline
    Single Event Snapback (SESB)              & high-current conditions                                              & N-channel MOSFET, SOI devices                                       \\ \hline
    Single Event Burnout (SEB)                & Destructive burnout due to\\ & high-current conditions                  & BJT Power MOSFET    \\ \hline
    Single Event Gate Rupture (SEGR)         & Rupture of gate dielectric due\\&to high electrical field\\ & conditions & Power MOSFETs \\ \hline
    
    \bottomrule
    
    \end{tabular}
    }
\end{table}



\subsection{Single Event Effects Mechanism}


When highly energetic particles, e.g., protons, neutrons, alpha particles passes through a semiconductor, it can directly or indirectly deposit charges in the silicon. This notion of deposit
charge is actually the generation of electron-hole pairs. Near the junction, these
pairs will first recombine at the polarized reverse PN junction.  Then quickly diffusion principle will prevail. SEEs are usually the profuct of the deposition and collection of charge over a sensitive node of the circuit~\citep{manuzzato2010single}. The duration of this
process is variable and can last from a few picoseconds to a hundred
nanoseconds.


\subsection{Basic FPGA Structure and SEE Effects on FPGA}



FPGAs are complex reconfigurable devices that comprise a wide family of different resources. The basic structure of modern FPGAs includes: interconnect resources, clock-management resources, configurable logic blocks (CLBs), input/output
blocks (IOBs), and embedded blocks such as digital signal processors (DSPs), general-purpose processors, high-speed IOBs, and memories. CLBs are used to perform simple
combinational and sequential logic. These blocks are typically formed of look-up tables
(LUTs), multiplexers, flip-flops, and carry logic. Programmable interconnect resources, such
as routing switches, allow interconnecting CLBs, IOBs and embedded blocks to implement multiple systems.
The logic and routing resources in an FPGA are controlled by the bits of a configuration memory, which may be based on either anti-fuse, flash, or SRAM technology. The
design flow of FPGA-based systems as shown in Figure~\ref{fig:fpga-struct} adapted from~\citep{hauck2010reconfigurable}.



\begin{figure}[tb!]
 \centering
  \captionsetup{justification=centering}    
   \includegraphics[scale=0.4]{figures/img/FPGA-structure.png}
   \caption{FPGA Structure and Design Flow~\citep{manuzzato2010single}}.
\label{fig:fpga-struct}
\end{figure}




The process starts with the design written
in a hardware description language (HDL), e.g., VHDL or Verilog. Next, the design is optimized and mapped into the FPGA’s available resources through logical synthesis,
technology mapping, placement, and routing. Finally, the generated bitstream downloaded into the device, and the device starts functioning according to the designer design.


Like any other semiconductor device, FPGAs are sensitive to radiation effects~\citep{hobeika2014multi}.
Mostly, these effects depend on the technology used to store the configuration data.
The foremost concern for SRAM-based FPGAs is
SEUs within the configuration memory, because the configuration memory controls all the operations, e.g., data and control~\citep{manuzzato2010single}.
Upset configuration bits may change the logic and routing of the implemented system, as
shown in Figure~\ref{fig:seu}, leading to functional failures in an unpredictable way. Anti-fuse and flash-based
FPGAs offer a relative immunity to SEEs, but these devices have a lower logic capacity, and anti-fuse is only time programmable, making SRAM-based FPGAs more
suitable for complex systems giving facility of more frequent reconfiguration and adaptation~\citep{quinn2015validation, violante2004simulation}. 

Errors produce in the FPGAs due to SEU can be classified into two different categories - errors affecting the logic blocks and errors affecting the routing~\citep{sterpone2006new}. For the logic block, the SEU can produce the following errors~\citep{sterpone2006new}.

\begin{itemize}
\item LUT error: SEU modified the bit of the LUT.

\item MUX errors: SEU changed the configuration of the MUX in a logic block.

\item Flip-Flop error: SEU modified the configuration of an FF.

Routing resources are about the 80 percent of FPGA resources; SEE can create different phenomena that modifies the programmable Interconnect Points (PIPs)~\citep{sterpone2006new}.

\end{itemize}



\begin{figure}
 \centering
  \captionsetup{justification=centering}    
   \includegraphics[scale=0.4]{figures/img/seu.png}
   \caption{Upset FPGA configuration bits may change the logic and routing~\citep{manuzzato2010single}}.
\label{fig:seu}
\end{figure}




%\subsection{Faults, and Failure}


\section{Design Verification by Fault Injection}




The second domain to understand in this project is designing, testing, and verification of the design (digital circuit) by fault injection on FPGAs. As we discussed before, SRAM-based FPGAs are particularly sensitive to SEUs. The configuration memory is the most sensitive part. By changing the configuration memory, may affect the overall functionality of the system. The work done so far to study the SEU effects on FPGAs, combines the emulation, simulation, and radiation testing.

\subsection{Emulation}

The purpose of fault emulation to reproduce as closely as possible radiation based results, evaluate the criticality of FPGA resources, analyze the impact of flipping the critical bits, and find the faulty response of the circuit. Several works have been done so far deals with the analysis and emulation of SEUs in an FPGA which is done by flipping the bit in the configuration memory. 

The work presented in~\cite{hobeika2014multi} provide an emulation platform for the signature generation. This work focused on the identification of the emulation zone, the fault list generation, the SEU emulation, and the result analysis. The purpose of their work is to investigate the sensitivity of SRAM-based FPGAs for evaluating the effects of SEUs. In this paper, they also provide the results for the radiation based signature and simulation-based signature. They showed that simulation and emulation-based signatures could contain the same error values as obtained with radiation but their probability of occurrence could significantly different. The arithmetic signature for TRIUMF to emulation is 85.3\% for adder and 84.8\% for the multiplier. 

The work presented in~\citep{souari2015optimization, souari2016towards} deals with fault injection into the FPGA configuration memory based on the sensitivity of the bits by considering that the "bits-at-one" are more sensitive to the "bits-at-zeros." They have developed a methodology to extract the list of "configuration bit addresses of the design LUTs" based on the essential bits file of the design named \textit{.ebc},  which can be extracted by the Xilinx \textit{BitGen} command.




An accelerated fault injection technique is presented in~\citep{di2014fault} based on the design essential bits provided by  \textit{bitgen}. Similarly, there are few other techniques availble for the fault emulation based on the random fault injection~\citep{faure2005single}.


The work proposed in the~\cite{hobeika2013flight} described a completed automated methodology to emulate SEUs on an FPGA efficiently. The authors used the reconfigurable flight control system as an application.
In this work, they are focused on the essential bits identification to speed-up the fault emulation and proposed on the new fault models.



The work presented in~\cite{quinn2015using} described the benchmark that can be used for the reliability and radiation effects study on FPGAs and microprocessors. 

%\begin{itemize}
%
%\item   {Classification of the configuration bits into subsets.
%        a.  Bits set to 1/0 of LUT.
%        b. Bits set to 1/0 configuring other than LUT.
%        c. Bits set to 1/0 configuring other resources not identified as  potentially critical by bitgen.}
%        
%        
%\item  {Estimating the number of critical bits of the set by randomly injecting faults in the bits of each set. This method helps to find the most critical zones of the FPGA.}
%
%\item {Prioritized the fault injection in the identified (step-2) most critical zones.
%These classification steps are done with the help of EBC and EBD files provided by the bitgen. The experimental results presented in [5] evaluated the SEU sensitiveness as well as bitgen efficiency. The results are evaluated between random fault injection with different prioritized bit subsets.  The first observation authors concluded - the bitgen did not accurately identify all the critical bits meaning the bitgen limitations. Second authors did the prioritizing the most sensitive subset. It would involve exhaustive fault injection. The authors used fault injection to get an estimated number of critical bits as well as the related estimation error. They used the term critical bit error estimate (CBEE). The authors claimed the CBEE observed for the random approach is higher than the observed under the bits subsets.  The ratio of observed critical bits (ROCB) observed for the random injection is far less than the different bits subsets.}



%\end{itemize}

\subsection{Simulation}

Simulation-based testing is low cost and flexible, but it is difficult to get the accurate results. For simulation-based testing, the work presented in the~\citep{violante2004simulation} described an approach for fault injection based on the simulation during early design phase when the hardware system is not ready.  This work helped to find the probability for an SEU to change the behavior of a given circuit.

The work presented in~\citep{robache2013methodology} demonstrates how faulty behavior --- signatures allow building high-level models, i.e., high-level faulty model in MATLAB simulink, that reflects the faulty behavior of a combinational circuit represented at gate-level. The fault injection tool used is named  LIFTING.  The purpose of this tool is to inject different types of faults on a circuit at gate-level. The tool used the stuck-at-0 and stuck-at-1 in each node of the design. The LIFTING is a simulation-based gate level fault injection tool that used the circuit netlist file, e.g.  *.v file. The work presented in this paper helps to make a faulty block with Simulink that reads a signature and generates errors according to the distribution.

\subsection{Radiation Testing}
Radiation testing is an expensive approach and requires a state-of-the-art facility. The work presented in~\cite{hobeika2014multi} described the effects of radiations on a circuit at the circuit-level. Authors compared the results which are expressed as signatures based on fault simulation, emulation, and radiation testing of the FPGA. They want to capture and regenerate the faulty behavior occurs due to SEUs early in the design process. 

The work presented in~\citep{dsilva2015neutron} used the Flash-based FPGA under neutron beam to observe the SEE in avionics applications. They observed;  that the failure-in-time rate for flip-flops, SRAM cell, PLL reasonably low. They proposed SEL testing procedure and observed that Flash-based FPGAs were immune to SEL. 

In~\citep{maillard2015neutron} authors tested the UltraScale Kintex FPGA under neutron and proton beam. They examined the single event upset response of the FPGA. They presented the result regarding a failure in time calculation. They observed that there were less than 0.1\% uncorrectable events. 




The work presented in~\citep{quinn2015using} described the software and hardware benchmark testing under the neutron test data. There is no standard test circuits available, researcher, used flip-flop or D-latches to compare their results. In recent years,  the research community has shown an interest to develop a standard set of circuits that include complex and realistic algorithms and can be adapted to different FPGAs. In this paper, authors presented a software and hardware benchmark for radiation testing. 


The hardware benchmark mentioned in this paper is ITC'99 which is well defined ATPG benchmark. The circuits are implemented in the HDL so that it can be ported to different FPGAs.  The radiation tests are completed at Los Almos Neutron Science
Center. The results are provided for the microcontroller, ARM cores, GPUs, and
FPGAs. The B13 from ITC99 and Virtex-5 is used
as a hardware platform. They also provide the results for the mitigation and unmitigated circuits. For mitigation, they
used X-TMR and VERI-Place. The failure-in-time (FIT) are decreased under mitigation,
but the overhead is increased.

Software benchmark radiation tests are done on the flash-based micro-controller. They also tested a ferroelectric-memory-based micro-controller, two ARMs, and GPUs. These components are tested with both mitigated and unmitigated codes. The results reported in the paper for two different microcontroller and two ARMs cores. For microprocessors: they observed the FITs are very small. In some cases, there is no error from the code during many days of testing. 










%\subsection{Benchmark for Radiation Testing}
%The suitable selection of the benchmark for the radiation testing of microprocessor and FPGAs is a recently topic of ongoing research. The benchmarks are used to evaluate the performance under different architectures, technology, and compiler. There is no such standard benchmark employed to study microprocessor and FPGAs under the effects of radiations; make it difficult to assess the changes in fabrication technology, architecture, and circuitry. 
%
%\begin{itemize}
%
%
%\item Homemade Design.
%\item Circuits from Opencore.
%\item Proprietary designs.
%\end{itemize}
%
%
%The problem with this approach as no two organizations used the same set of codes or circuits, difficult to make the comparison. There is a need for collaboration to make a suitable set of benchmark for reliability application and study the effects of radiation under the same conditions. The criteria used to set a standard benchmark including:
%
%Repeatability of benchmark tests.
%A representative of deployed computing workload.
%Availability of fixed input vectors.
%Cross-platform implementation.
%The ability to repeat test itself is an important part of the standardized testing. By repeating the algorithms, the input test vector, the compilation, the synthesis setting help researchers to have the enough information. It is necessary to provide a wide variety of realistic algorithms so that the system can be tested as likely to the realistic application. Defining the input test vector is an essential step because many hardware errors can be observed under the specific set of the test vector. It is an open question which input test vector should be adopted, under the specific set of criteria. Finally, the implementation of the algorithms in portable languages help to use the same set of codes on the different platform. For example, assembly language for the microprocessors limit the ability to compare and port codes on the different platform. But the hardware benchmark developed in VHDL can ease the problem; the same circuit can be ported to any FPGA.
%
%\textbf{FPGA Radiation Benchmark}
%
%T

%\textbf{Software Radiation Benchmark}
%
%The software radiation benchmark is harder to design than the FPGA radiation benchmark. The development of the standard set of algorithm that can be ported on different architectures would be a challenging task e.g., porting an algorithm to 16-bit microcontroller to GPU. The authors are interested in the software benchmark where the computational load can be divided into the parallel processes or run on a single core. The commonly used software benchmark comprises of fast fourier transform, matrix multiplication and quick-sort algorithm as they are commonly used in many applications and useful for the evaluating the reliability of parallel processors. The software benchmark comprises the following code.
%
%\begin{itemize}
%\item AES-128;
%\item Cache test;
%\item FFT;
%\item Hotspot;
%\item HPCCG;
%\item Matrix Multiply;
%\item Quicksort
%\end{itemize}










%\section{Fault-detection, mitigation and correction in the FPGA }


%The impact of SEUs on SRAM FPGA devices has been studied in~\cite{bellato2004evaluating}. Many  techniques  have  been  proposed to provide highly reliable FPGA devices, e.g. radiation-hardened FPGAs~\cite{rockett2007radiation}, in-order to lower the effect of radiation-induced SEUs. However, radiation-hardened  SRAM  FPGAs  typically have  a  low  density, and  they  only  may  lower  the  probability of SEUs to occur but  not  completely avoid  them. Therefore, non radiation-hardened FPGAs, like the  Xilinx Kintex-7, are evaluated under a harsh radiation  environment~\cite{wirthlin2014soft}. Even on radiation-hardened FPGAs, the SEU rate in a low-earth orbit flight experiment can be up to 16 events per day~\cite{quinn2012orbit}. A wide  variety  of  SEU  fault  mitigation  techniques  for SRAM-based  FPGAs  have  been  proposed  during  the  past years. These techniques can be categorized into module redundancy techniques such as triple modular redundancy (TMR)~\cite{lyons1962use} and techniques that use scrubbing of the FPGA configuration memory~\cite{heiner2009fpga}. Also the combination of  both techniques has been shown to be able to increase the reliability of FPGA modules significantly ~\cite{ostler2009sram}. FPGA-based TMR approaches replicate a given module which shall be protected either statically or dynamically~\cite{angermeier2011runtime}. The different granularities of voted replicas  are evaluated in~\cite{bolchini2007tmr}. However, no upset rates and consequential no reliability figures are provided. Nevertheless, TMR techniques are  known to often cause an excessive and unacceptable overhead in terms of power  consumption and area. Since the intensity of a cosmic rays is not constant but may vary over several magnitudes depending on the solar activity, a worst-case radiation protection is far too expensive in most cases. A self-adaptive system is proposed in~\cite{glein2014self}, which monitors the current SEU rate and exploits the opportunity of partial reconfiguration of FPGAs to implement redundancy such as TMR on demand. 

%Memory scrubbing is a well-known correction technique for the configuration memory of SRAM-based FPGAs. It consists on re-writing the configuration memory after the FPGA is configured to restore its original content. It is often a transparent operation for the running application. This is possible because modern FPGAs offer a dynamic partial reconfiguration (DPR) feature. The circuit that enables the scrubbing is commonly named scrubber. Additionally, readback is the process of reading the configuration memory of the FPGA after it is configured. Both processes (readback and scrubbing) can be used to implement different scrubbing methodologies as shown in~\cite{herrera2013design}. Scrubbing can be implemented using an internal or external interface as shown in~\cite{berg2008effectiveness}. When external interface is used, the scrubbing logic is implemented outside the FPGA. In the case of Xilinx FPGAs several external interfaces are available; however, the Select MAP interface has the highest data throughput. On the other hand, there is only one internal interface named ICAP~\cite{xilinx}. This internal interface can be accessed from the reconfigurable logic of the FPGA and it is a replica of the Select MAP interface. Also scrubbers can be implemented in software or hardware. The scrubbing process can be implemented using a microprocessor with the advantage of a high flexibility to implement different complex scrubbing methodologies but with lower configuration speeds and lower energy efficiency.



\section{Fault Models}


The last domain of this work is related to the fault models, and fault analysis. We further divided into three different categories: behavior domain, transformation and circuit domain and provide a related work on each domain separately.


\subsection{Behavioral Domain:}




In~\citep{svenningsson2010model}, the authors showed how model-based fault injection could be utilized to
simulate the effect of hardware related faults in embedded systems. Model-level and hardware-level
fault behavior comparison is presented. They performed the emulation on the micro-controller. They developed a tool which changed the bit value in the register or memory cell through assembly language program. They modified the code through assembly tools and analyzed the behavior of the fault.

In~\citep{hayne1999behavioral}, the authors proposed two tools, which facilitate the fault simulation of behavioral
models, described using VHDL. The tool is the Behavioral Fault Mapper (BFM). The BFM algorithm
accepts a fault-free VHDL model of the design (combinational circuit) and a fault list of N faults from
which it produces N faulty models. The assumed eight different fault models, e.g., Stuck-Then, Stuck-Else,
Assignment Control, Dead Process, Dead Clause, Micro-operation, Local Stuck-data, and Global Stuck-data. 

In~\citep{chen2017fault}, the authors proposed fault propagation process between the different subsystems of the
main system, combined with the finite state machines. This work is based on the FSM and fault propagation models. In this work, authors analyzed how the fault comes in one sub-system effects the others and leads towards the overall failure of the system. They calculated the Mean Time Between Failure (MTBF).


The work presented in ~\citep{mirzadeh2014modeling} proposed a fault behavior model developed with a neural network
concept. The neural network is used to synthesize the faulty output of a
circuit at a high-level of abstraction. They used a neural network to replicate the faulty behavior of the circuit in the presence of the fault. The idea is good to make a model with neural networks. But the work is done based on simulations only, no hardware experimental results reported. Moreover, the author claimed in their "proposed methodology chapter"  --- "The goal of developing a behavioral fault model of a circuit is to create a library of faulty components reusable at high-level of abstraction." But they didn't provide any results for the library of faulty components.


In~\citep{janschek2017errorsim} authors proposed the high-level fault behavior model. This work used for the error propagation analysis. They developed a tool that has the ability to simulate different fault-models, e.g., offset, stuck-at-fault, noise, delay, package drop, and bit-flip. They used the passenger jet Simulink model for the experimental purpose. They analyzed the critical and non-critical sensor fault for the closed-loop passenger jet control sequence.

In~\citep{hobeika2013flight} authors presented the new fault model that can be used by the designer at an earlier stage in the design process. In this work, they used the adaptive control model and did the SEU emulation on it. They found the new fault model, e.g., control amplification, sign inversion, etc. They also provide the information about the design essential bits and the bits used for the fault emulation.

The work presented in~\citep{thibeault2013library} based on the C/C++ description of an application. They used the technique to convert the circuit into control and data flow graph (CDFG) file, then used the resource estimation tool, to find the resources required to implement an application on an FPGA. This work is based on the .xdl and .ncd file of the design provided by the Xilinx tools, which are no more compatible with new FPGAs.


The work presented at the behavioral level did not have the insight information of the how fault influences the hardware at the circuit level. For example, the work presented in~\citep{janschek2017errorsim} supposed the fault distribution is normal, exponential, Poisson, Weibull. The most of the research work has been done is related to the transient fault analysis, soft error rate (SER) analysis and error prediction, error propagation,and failure-in-time calculations.


The methodology we present in this thesis based on the fault emulation at the circuit-level and abstract it to the high-level. Details are in the Chapter~\ref{approach}, in which detail information given about the model abstraction at the high-level based on the low-level circuit behavior.


\subsection{Transformation Domain}



There has been a lot of work done for the analysis of the faulty circuits and error propagation based on the techniques that involve the transformation of the circuit or design into some other equivalent domain, e.g., Binary Decision Diagram, then analyze the fault and error behavior. Compute the SER  and error probabilities.

The work presented in~\citep{ubar2014modeling} is based on the simulation of the faulty circuit.  The authors transformed the circuit into their required tool format, i.e., structurally synthesized BDD, then they
insert the fault on different nodes to calculate the SER. 


Modeling the probabilistic behavior of the Finite State Machine, calculating the steady state
behavior of the circuit and used for estimating the switching activity of the circuit for
power evaluation presented in~\citep{hachtel1996markovian}. They showed how steady-state probabilities of very large FSM’s could be computed by
symbolic ADD-based algorithms. In this work, authors need to convert the FSM into respective BDD model to perform the fault simulation.


In~\citep{shazli2011high}, soft error rate  computation problem is modeled as a Boolean
Satisfiability (SAT) problem and SAT solvers are used to compute SER for combinational and
sequential circuits. They used an automated flow to convert combinational and sequential behavioral
descriptions into equivalent SAT instances and analyzed them for SER.



In~\citep{miskov2007mars}, the symbolic framework based on Binary Decision Diagram / Algebraic Decision
Diagram is presented for sequential circuit analysis. They determined output gate susceptibility to error. The error is calculated at the transistor level. They also estimate the gate error probabilities.


The limitation of working in this domain require converting the circuit into a tool with many assumptions. Most of the work is based on the mathematical and analytical expressions. This requires deriving the mathematical expressions for the fault model in the tool converted format. This work also needs to assume the fault will create an erroneous output, plus its too hard for complex circuits to transform and derive their mathematical expressions.


\subsection{Circuit Level}


In~\citep{li2016monte} authors presented a  Monte Carlo technique for the soft error analysis of the sequential
circuits. They performed the logic simulation for latch-level error propagation to estimate the SEUs. This work is mainly focused on the SEUs convergence meaning after how many clock cycles the circuit becomes fault free.
They insert the fault into the simulator and observe it to the output. They apply the Monte Carol technique
to find the number of samples and the time for the estimation of the error.
  
  

The key idea behind~\citep{ebrahimi2015comprehensive} is to present the result for SER analysis on an embedded
processor. Their platform employs a combination of models at a device level, a gate level, and architectural level. They used the
processor core and derived all the results via simulation under the assumption of the fault. While using processor core under the radiations, the primary concerns is for the cache memory instead
of the core itself.


The technique used in~\citep{ranjan2014aslan} is to estimate the output quality of a faulty circuit. They used the concept
of unrolling the circuit with each clock cycle and compares faulty unrolled circuit with the original circuit and find the errors.
The cost of unrolling and analyzing error metric also increase
with circuit complexity.

In~\citep{yu2010scalable}, authors developed novel and efficient ways to examine a behavior of a circuit. They model the impact of soft errors and estimate the circuit reliability. Their method is based on the probabilistic transfer matrices to calculate signal and error probability distributions in the
sequential circuits at the logic level. This work is mainly focused on finding the signal probabilities at the gate level. They partitioned the combinational part of the sequential circuit, and then find the probabilities of each gate.


In~\citep{miskov2007mars},  authors estimated the likelihood that a SET in a sequential circuit will lead to errors
in clock cycles following the particle hit, and found after the hit (which is an assumed fault) how many clock
cycles needs to get the SER below the threshold level which they set. The main idea is to do symbolic modeling and efficient estimation of the susceptibility of a sequential circuit to soft errors.




In~\citep{lingasubramanian2010probabilistic}, authors calculated the maximum error in digital circuits and found the respective
worst-case input pattern, through a posteriori hypothesis, using a Shenoy-Shafer algorithm.
They showed the importance of handling maximum error behavior for achieving fault tolerant
computing machines. .

In~\citep{miskov2008modeling}, Markov chain for the steady-state behavior of the sequential circuits, a Symbolic framework
based on the BDD/ADD is presented. They did the Single Error Rate evaluation, purposed to do
for the gate sizing, find the gate size that has the highest soft error impact based on this recommend
different gate size for the transistor technology. 
 




In~\citep{das2007monitoring}, error monitoring scheme to detect the transient error is presented. The idea presented in this work has worth to compute the error for each stage
independently. But they didn't provide they experimental setup details, to analyze the benefits and overhead of this technique.

In~\citep{asadi2005soft}, they provided the multi-cycle analytical framework to analyzed the multi-cycle error
propagation. The major limitation of their work is the measuring unit they used  “mean time
to manifest error.” The standard terminology for this kind of work is SER. This is the simulation work,
with the assumption that the fault will occur in the flip-flop and proceed to another flip-flop. They did not
consider the fault occurred in the combinational logic of the sequential circuit.

In~\citep{miskov2010multiple}, authors presented an idea to analysis the susceptibility of the circuits. The analyzed the outputs
errors originating from the single or multiple fault transients. They are keener to find the part of
the circuit that has the highest error generating probability.







The work is done in this domain used simulation setup for error calculation, assumed fault occurred, supposed the glitch size, and assigned the
probabilities for fault propagation in the HSPICE simulator. Authors make the model of the faulty circuits, analyzed them using modeling techniques, compare results with the simulations results derived by using the
HSPICE. 

They also assumed that an error occurs only in the first clock cycle of a w-cycle simulation with
no new errors occurring in subsequent cycles which also underestimates the usability of this work.  There is one more bottleneck if the circuit size and the number of primary
inputs and outputs grow linearly with the number of simulated cycles, and memory usage becomes
unmanageable after a few simulated cycles.










\label{related}





%%% Local Variables:
%%% mode: latex
%%% TeX-master: "../Document"
%%% End:


\chapter{Proposed Approach}
%http://web.mit.edu/course/21/21.guide/capitals.htm
This chapter is dedicated to the methodology that we propose to achieve the objectives of this thesis, i.e. Methodology and Algorithms for High-level Modeling of CR Impacts on Electrical Systems.




The work accomplished so far in the literature studied fault behaviors at high-level of abstraction under the assumption of fault occurrence. As discussed in Chapter~\ref{faultmodels} there is no evidence available, are these models exists for an FPGA based system? If yes, What is the probability of observing any particular fault model? How many different fault behaviors occur under the radiation experiment that lasts few hours?  What is the relationship of these faults models with the underlying hardware architecture regarding bitflips?

We would like to answer all these questions by making the HMM. The HMM helps to find the answers of all the questions mentioned above, once we get the answers of these questions, designer no longer needs to do the hardware-based experiments to find the fault behavior and particular hardware resource utilization. 

A designer can find the answer by giving the fault models, its probability of occurrence, original circuit to produce faulty components library, respective VHDL component to find resource utilization. I would like to start by giving the fault models we may get during the testing, then explain how I use HMM to solve our problem. How to make a faulty components library, its utilization will be discussed at the end of this Chapter.
\label{approach}
\section{Relation to State-of-the-Art}
Starting from the background, remember the facts established in the Chapter~\ref{introduction} mainly the fact that the digital circuits vulnerable to radiation require high-reliability requirements. The faults e.g., stuck-at-fault due to particle, strikes cause different effects in the system behavior based on an FPGA. As discussed in Chapter~\ref{related} in digital sequential circuit, a single event upset causes a multiple faulty response of the underlying circuit. Consider the example of  a 3-bit counter as shown in Figure~\ref{fig:counter}. The node is stuck-at-0$\rightarrow C_0$. This stuck node response can propagate to multiple outputs, as shown in Table~\ref{c@0-c0} and produce multiple erroneous outputs. The way this SEU interrupt at high-level of abstraction could typically manifest multiple correlated; "change in states of the design," and "change in signature values." To model the fault occurs at the low-level, and make the model of the faulty response of the circuit at high-level of abstraction, I propose to use the Hidden Markov Model. 




\begin{figure}[tb!]
 \centering
  \captionsetup{justification=centering}    
   \includegraphics[scale=0.6]{Figures/counter.pdf}
   \caption{Signature Generation.}
\label{fig:counter}
\end{figure}
\begin{table}[tb!]
\center
\caption{Stuck-at-0$\rightarrow C_0$}
\label{c@0-c0}
\begin{tabular}{|c | c| c | c| } 
 \hline
 \rowcolor{lightgray}
Faulty Value(Binary) & Faulty Value & Original Value & Arithmetic Signature   \\ 
\hline
 
 
 000& 0 &0 & 0  \\
 \hline
 001 & 1 & 1 & 0 \\ 
 \hline
 
 000 & 0 & 2 & 2 \\
 \hline
 001& 1& 3& 2 \\
 \hline
 000 & 0  &  4& 4 \\
 \hline
 001 & 1 & 5 &4  \\
 \hline
 000 & 0 & 6 & 6 \\
 \hline
 001 & 1 & 7 & 6 \\
 \hline
 
 
\end{tabular}
\end{table}
\section{Description of Fault Behavior}
Before going into the details of the HMM approach, I prefer to give a brief description of fault behaviors that we may observe during the radiation testing and fault emulation. Based on the fault models mentioned in the Chapter~\ref{faultmodels} the most prominent fault behavior we will observe is "Stuck-at-fault." However, there are few other fault behaviors; we might experience which shows the limitation of the stuck-at-fault model, because these fault behaviors cannot be imitate with stuck-at-fault. We need to develop more fault models at high-level to imitate the same faulty behavior. The pseudo-codes for some high-level fault models are mentioned in the Appendix~\ref{appendix}.
\begin{itemize}
\item \textbf{Lost signal or event fault:} This kind of fault model we might observe when a certain process of a system stop working. For example, a 3-bit counter, the result of the first two bits are accurate but the third flip-flop didn't respond.

\item \textbf{Stuck Signal:} In this kind of fault model the signal is stuck to some fixed value. For example, the input of a FIR filter stuck to some unknown fixed value. 
 
\item \textbf{Variable change fault:} This fault behavior shows the situation in which either inputs, outputs, or intermediate calculations changes to some other value. Now, consider a case as shown in Table~\ref{FMFIR}, the expected faulty output of the FIR filter: $\{1, 6, 5, 16, 15\}$ due to variable change fault the faulty output observed: $\{1, 80, 5, 16, 15\}$. The signature observed from the variable change fault is only one position different from the expected signature (stuck-at-fault).  It means we cannot use the stuck-at-fault model to imitate this kind of faulty behavior. This is  the limitation of the stuck-at-fault (applies to other models as well). In this thesis will make different fault-models to represent that signature.
\begin{table}[tb!]
\center
\caption{Variable Change Fault Models FIR}
\label{FMFIR}
\scalebox{0.75}{
\begin{tabular}{|c|c|c|c|} 
 \hline
 \rowcolor{lightgray}
\shortstack {Faulty Value \\(Stuck-at-fault)} & \shortstack {Faulty Value \\ (Variable change)} & \shortstack {Signature \\ (Stuck-at-fault)} & \shortstack {Signature \\ (Variable Change)}   \\ 
\hline
 
 
 1& 1 & 0 & 0  \\
 \hline
 6 & 80 & 1 & -75 \\ 
 \hline
 
 5 & 5 & 0 & 0 \\
 \hline
 6& 16& 1& 1 \\
 \hline
 15 & 15  &  0& 0 \\
 \hline
 
 
\end{tabular}
}
\end{table}
\item \textbf{Changing the specified range of output:} This kind of faulty behavior shows if we get the output which is out of the range of the expected value. For example, the 3-bit counter gives the output that is out of the range of 3-bit counter, e,g., $out$ $<=$ $10$.
\item \textbf{Delayed Fault:} This fault model explains the situation in which the output is delayed according to the predetermined outputs. For example, the output of the counter comes after ten clock cycle delays.
\item \textbf{Invert Fault:} When the fault occurs, and it inverts the actual value, e.g., true to false and false to true. This kind of fault observes for \textit{if} and \textit{else}
conditional statements. For example, the \textit{if} statement start working on condition: \textit{false} instead of \textit{true}.

\item \textbf{Short Circuit Fault:} The fault model makes the two independent variables dependent on each other.

\item \textbf{Open Circuit Fault:} The signal can take any non-deterministic value.

\item \textbf{Stuck-Then:} This fault model describes the failure of the \textit{if-then-else} statement in which the \textit{else}
 statement never executed. In the following example, the \textit{else }
statement never execute.

\item \textbf{Stuck-Else:} This fault model represents the behavior in which the \textit{then} statement never execute.

\item \textbf{Dead Process:} This fault model describes the failure of the statement inside the \textit{process} statement.

\item \textbf{Micro-operation fault}: This fault model represents the failure of the operation of the logical operator, arithmetic operator, and relational operator, e.g., xor <=.   
\item \textbf{Swap Value Fault:} The fault model depicts swapping the values between two variables.

\item \textbf{Constant Fault}: This fault model force the variable to assign some unknown constant.

\item \textbf{Amplification}: This fault model force the variable to multiply with some value.

\item \textbf{Increase and Decrease }: This fault model force the variable either increase or decrease from its original value.
\end{itemize}



\section{Why Hidden Markov Model?}
We plan to use HMM for modeling and analysis of the sequential circuits susceptible to soft-errors. We prefer to use the HMM analysis over the other modeling techniques, e.g., BDD/ADD, Boolean Satisfiability problem (SAT), Monte-Carlo sampling, approximate approaches, symbolic methods for efficient estimation, and simulation methods. As all these techniques require to transform the circuit into their respective tool and require a lot of mathematical notations. While in HMM we can directly make the model just looking into the signature values, construct the high-level model that can be further use for the analysis. HMM can be used suitably to model systems which consist of different observable outputs (signature in our problem domain) on different hidden conditions, i.e., fault behavior, e.g., stuck-at-fault, or delay fault.
To make the HMM, we first need to have two copies of the original circuit --- named Fault-Free circuit, and a faulty circuit (\textit{hit circuit}). The Fault free circuit is used to collect the correct behavior of the circuit (fault-free outputs), and faulty is used to collect the faulty response of the circuit (faulty output). The difference between them gives us the signature as shown in Figure~\ref{fig:SG} and represented in the following equation.
\begin{center}
$Arithmetic$ $Signature$ $=$ $Original$ $Value$ $-$ $Faulty$ $Value$
\end{center}
 \begin{figure}[tb!]
 \centering
  \captionsetup{justification=centering}    
   \includegraphics[scale=0.8]{Figures/signature1.pdf}
   \caption{Signature Generation}
\label{fig:SG}
\end{figure}

Hidden Markov Model is a statistical model in which a system that can be modeled is assumed to be a Markov process with unobserved states, i.e., hidden states. A Markov process  is a stochastic process that satisfies the Markov property -- \underline{memorylessness}, meaning, a process that satisfies the Markov property, if the prediction of the future of the system's output based solely on its present state, it is independent of the future and past states. In order to apply the HMM, we need a system that generates probabilistic output patterns in time, e.g., faulty response of the 3-bit counter. Afterwards, we need to look at the system and need to know which states of the system give that particular output ---  the underlying system is hidden. For example, in the case of a 3-bit counter, the observed sequence is the "signature" and the hidden is the "stuck-at-fault or any other fault." Now, we wish to devise a model to predict that "fault behavior", without actually knowing about the fault. 

In the 3-bit counter example we  get   different signatures. Here for simplicity, I give the example of four different signatures and construct the HMM. The signatures are: \textit{Sign-1, Sign-2, Sign-3, and Sign-4}. The observed signatures are probabilistically related to the hidden process. We can model such process using a Hidden Markov model, where there is an underlying Hidden Markov process changing over time, and a set of observable states which are related to somehow to the Hidden states.
The figure~\ref{fig:HMM-3-bit} shows the Markov model for the hidden and the observable states for the 3-bit counter example. The hidden states are (stuck-at-fault, delay fault, and variable change), and observable states are  the signatures. 
\begin{figure}[tb!]
 \centering
  \captionsetup{justification=centering}    
   \includegraphics[scale=0.8]{Figures/HMM.pdf}
   \caption{HMM model 3-bit counter.}
\label{fig:HMM-3-bit}
\end{figure}

HMM is based on  two things: a) Observable States, and b) Hidden States. If we closely observe above mentioned 3-bit counter example, we realized that in this scenario --  signatures are the observable state, and the faults occurs due to the bit-flip are the hidden states. In this thesis, we used the concept of HMM to construct the high-level model based on the signature information.

\section{Modeling Hidden Markov Model}
\begin{figure}[tb!]
 \centering
  \captionsetup{justification=centering}    
   \includegraphics[scale=0.8]{Figures/HMM-air.pdf}
   \caption{Hidden Markov Model to the FPGA based fault emulation system}
\label{fig:HMM-air}
\end{figure}
The Figure~\ref{fig:HMM-air} shows the basic HMM applied to the FPGA based fault emulation system. We can model this system with the HMM, where, the outputs emitted by the system (FPGA under radiation/ Fault Emulation) are observable (signatures --- in our problem domain), and underlying states of the system (Stuck-at-fault, delay fault, swap value fault, etc.). HMM can be visualized as a simple finite state machine. The HMM has a strong statistical foundation; it has the ability for efficient learning algorithms, which can take place directly from the raw sequence data. \textbf{The problem in hand} can be solved by using the HMM as we can observe the sequence of signatures, but we do not know  which fault behavior  of the system it went through to generate that particular signature. Because we don't know which node is attached by the SEU during the radiation experiment or the fault injection. The analyses of Hidden Markov Model seek to recover the states from the observed data.


\subsection{Probabilities in a HMM}
There are three important things to know about the probabilities in HMM.
\begin{itemize}
\item The connections between the hidden states of the system, and the observable states of the system represent the probability of generating a particular observed state given that the Markov process is in a particular hidden state as shown in Figure~\ref{fig:HMM-3-bit}.
\item The probabilities entering into the observable states will sum to "1." 
\begin{center}
$Pr(Sign-1|Stuck) + Pr(Sign-1|delay) + Pr(Sign-1|variable $ $ change)  = 1 $
\end{center}
\item In addition, probabilities define the Markov process, we have another matrix termed as "confusion matrix", which contains the probabilities of the observable states given a particular hidden state. The following matrix is the confusion matrix for the 3-bit counter example. We can extract the confusion matrix values from the radiation and fault emulation experiment.
\begin{center}
CM = \bordermatrix{~& Sign-1 & Sign-2 & Sign-3 & Sign-4\cr
                  Stuck@0\rightarrow{C_0}& 0.60 & 0.20 & 0.15&0.05 \cr
                  $Delay$   & 0.25 & 0.25 &0.25 &0.25\cr
                  $Variable$ $ $ $Change$ & 0.05 & 0.10 &0.35 &0.50\cr}
\end{center}
\end{itemize}
\subsection{HMM Parameters}
A hidden Markov Model is described as $ \Pi = \{\pi, A, B\}$ as shown in Figure~\ref{fig:MARKOV-SERIS}, where,
\begin{figure}[tb!]
 \centering
  \captionsetup{justification=centering}    
   \includegraphics[scale=0.8]{Figures/MARKOV-SERIS.pdf}
   \caption{HMM Formalization.}
\label{fig:MARKOV-SERIS}
\end{figure}
 $(\pi) = $ initial state probabilities vector; 

$A = (a_ij)$ state transition matrix;  \hspace{0.3cm} $P_r(x_i | {x_j}_{t-1})$


$B = (b_ij)$  confusion matrix;     \hspace{0.3cm}        $P_r(y_i | x_j)$ 


We can find the initial state probability vector, state transition matrix and confusion matrix  from our experimental data generated by the radiation or emulation experiment.
\textbf{Hidden States:} The true states of the system that may be described by a Markov Process, e.g., stuck-at-fault, etc.,
\textbf{Observable State:} The state of the process, visible, e.g., Signature.
\textbf{$\Pi$ $Vector$:} contains the probability of the hidden model being in a particular hidden state (fault behavior) at the time t = 1.
\textbf{State transition matrix:}  holding the probability of a hidden state given the previous hidden state.
\textbf{Confusion matrix:} containing the probability of observing a particular observable state given that the hidden model is in a particular hidden state. 
Once we model a system with HMM, it helps to find three problems. We model them according to our problem, i.e.,
\begin{tcolorbox}[width=\textwidth,colback={gray},title={Evaluation },colbacktitle=gray,coltitle=black]  
Find the probability of an observed signature given an HMM.  
\end{tcolorbox}
\begin{tcolorbox}[width=\textwidth,colback={gray},title={Decoding },colbacktitle=gray,coltitle=black]  
Finding the hidden states (fault models) that most probably generated an observed sequence. 
\end{tcolorbox}
\begin{tcolorbox}[width=\textwidth,colback={gray},title={Learning },colbacktitle=gray,coltitle=black]  
The third problem is generating an optimized HMM given a sequence of signatures (observations).
\end{tcolorbox}
\section{HMM Application for Signature}
Hidden Markov Model can give the answer to three major questions that we can use in our problem domain. It helps to a) compute the probability of a given sequence of signatures, b) compute the most probable sequence of states, and c) given a sequence of observations and learn the best HMM model.
In order to find the solution of these three questions, we called it HMM applications.
\textbf{The Three basic HMM Applications are:}
\begin{itemize}
\item Application 1 : Probability Evaluation
 \begin{itemize}
 \item How do we efficiently compute $P(O|\Pi)$ the probability of the signatures given that the HMM parameters from the given observed signature sequence $O = {O_1, O_2,...,O_n}$.
 
  \begin{itemize}
  \item We can use the forward algorithm~\cite{ghahramani1996factorial}.
  \end{itemize}
 \end{itemize}
\end{itemize}
\begin{itemize}
\item Application 2 : Optimal State Sequence
 \begin{itemize}
 \item Given signatures sequence $O = {O_1, O_2, ...., }O_n$ and model $\Pi$ how we choose a hidden state sequence (stuck-at-fault, variable change fault, etc.) $Q={q_1,q_2,q_3, ..., q_n}$
that is optimal, i.e., best explains the data. 
  \begin{itemize}
  \item Viterbi algorithm~\cite{forney1973viterbi} helps to find the answer to this application.
  \end{itemize}
 \end{itemize}
\end{itemize}
\begin{itemize}
\item Application 3 : How do we adjust the parameters of the model $\Pi = \{\pi, A, B\}$ to maximize the likelihood $P(O|\Pi)$ 
 \begin{itemize}
 \item Given observation sequence $O = {O_1, O_2,....,} O_n$ adjusting the hidden HMM parameters $ \Pi = \{\pi,A, B\}$ to maximize the probability $P(O|\Pi)$ 
  \begin{itemize}
  \item The solution is either to use Expectation-Maximization~\cite{moon1996expectation} or Baum-Welch’s re-estimation~\cite{leggetter1995maximum}.
  \end{itemize}
 \end{itemize}
\end{itemize}


\subsection{Evaluation Application}
For probability evaluation, we need to compute the likelihood of an observed signature sequence $O = {O_1, O_2,...,O_t}$ given a particular HMM model $ \Pi = {\pi, A, B}$. The computation of this probability involves all the possible hidden state sequence and evaluate the corresponding probability. 
\begin{itemize}
\item  $P(O | \Pi) = \sum\limits_{\forall Q}^{} P (O | X, \Pi) P (X, \pi)$ 
\item For a specific state sequence $X = {x_1, x_2,...,x_t}, P(O | Q, \Pi):$


 $P (O | X, \Pi) = \prod_{t=1}^{T} P (o_t | q_t, \Pi) = \prod_{t=1^{T} b_{x_t} (o_t)}$
 
 \item The probability of the state sequence $X$:
 \\
 \hspace {0.2cm} $ P (X | \Pi ) = \pi_{x_1} a_{x_1 x_2} a_{x_2 x_3},...,a_{x_{T-1} x_T}$
 
 \item The final expression we get:
 
$P (O | \Pi ) = \sum\limits_{x_1, x_2,..., x_T} \pi_{x_1} b_{x_1} (o_{x_1}) a_{x_1 x_2} b_{q_2} (o_{x_2}),..., a_{x_{T-1} x_T} b_{xT} (o_{xT})$
\item If there are $N^T$ possible state sequence, this approach becomes infeasible to apply or implement even for the smallest circuits.
\begin{itemize}
\item For N = 5 and T = 100, the order of magnitude --- $10^7$
\end{itemize}
 
\end{itemize}
This problem can be solved by using the Forward Algorithm:
\textbf{Example:} Consider an example where we have a number of HMMs (a set of triplets $(\pi, A, B)$) describing different systems, and a sequence of observation, i.e., you will get this system by performing radiation/fault emulation experiments for hours of testing. You have a number of HMMs constructed in the Simulink library and want to know which HMM most probably generated given sequence (signature).
We will use the forward algorithm to calculate the probability of an observation sequence given a particular Hidden Markov Model, and find the most probable HMM. Suppose that you have an HMM that describes the fault behavior, and we also have a sequence of signatures. Suppose the fault behavior works in this order $(stuck-at-c_0)$, delay fault, dead process, the signatures: Sign-1, Sign-2, and Sign-3. There is some hidden relationship between fault behavior and the signatures, we can make a "Trellis" diagram as shown in Figure~\ref{fig:trellis}
\begin{figure}[tb!]
 \centering
  \captionsetup{justification=centering}    
   \includegraphics[scale=0.8]{Figures/viterbi.pdf}
   \caption{Trellis Diagram}
\label{fig:trellis}
\end{figure}
From the trellis figure we conclude the following:
\begin{itemize}
\item Each column in the trellis represents the possible state of the fault behavior and each state in the column is connected to the each state in the adjacent column.
\item The transition between the states --- state transitions has the probability provided by the "state transition matrix."
\item Each column in the signature observations at that time: the probability of this signature observation given any one of the above fault behavior states is provided by the confusion matrix. 
\item As mentioned above one of the possibilities of calculating the probabilities of the observed states would be finding each possible sequence of the hidden fault behavior states and sum all these probabilities. Just for this example, there would be $3^3 = 27$ possible sequences, it’s extremely complex to do this. So, we will propose to use the forward algorithm that can calculate the probabilities of observing a sequence recursively given HMM.
\end{itemize}
\textbf{Forward algorithm Steps:}
We need to calculate the probability of observing a signature recursively given HMM. 
\begin{enumerate}
\item The first step is the initialization step at t = 1 when there is no path to the state. The probability of being at state at t = 1 is actually the initial probability:
\begin{itemize}
 \item $P (state | t = 1) = \pi $
\item The initial probability at t = 1 is the probability multiplied by the associated observation probability.
$\alpha(j) = \pi(j) b_i (O_1)$
\end{itemize}
\item Second, we need to define a \textbf{partial probability} which is the probability of reaching an intermediate state in the trellis.
\begin{itemize}
\item For example, the T-long signature sequence: $(Y_k1, Y_k2,..., Y_kT)$, partial probabilities $\alpha 's$. Figure~\ref{fig:trellis} shows the Trellis diagram. Calculate the probability of reaching an intermediate state in the trellis diagram as the sum of all possible paths to that state.
\item The partial probability of state $j$ at time $t$ is $t(j)$, to calculate the partial probability:
$\alpha_t(j) = P (observation $ $ | $ $ hidden state $ $ is $ $ j) \times P (all $ $ paths $ $ to $ $ state  $ $ j $ $ at $ $ time $ $ t)$
\item The partial probabilities for the final observation hold the probability of reaching those states going through all the possible paths. The sum of these final partial probabilities is the sum of all possible paths. 
\hspace {4.5 cm}

$\alpha_{t+1}(j) = b_jk_{t+1} \sum\limits^{n}_{i = 1} \alpha_t(i) a_{ij}$

\end{itemize}
\item This expression can be used to calculate the $\alpha$. We can find the probability of an observation given HMM. The probability of the sequence given the HMM is then the sum of the partial probabilities at time t = T.
 
 \hspace {4.5 cm}$P(Y^K) = \sum\limits^{n}_{j=1} \alpha_T (j)$
\end{enumerate}

\subsection{Decoding Application}
\begin{figure}[tb!]
 \centering
  \captionsetup{justification=centering}    
   \includegraphics[scale=0.8]{Figures/fromhiddentosignature.pdf}
   \caption{Fault behavior to Signatures}
\label{fig:HMMsig}
\end{figure}
To design it for the decoding application there are two possibilities to use either Viterbi decoding algorithm or  posterior decoding. The Viterbi gives the most likely sequence while posterior decoding gives the most likely state at each position. Here, we are more focused on the sequence of hidden states (use in simulators to reproduce the signatures) that give the respective signatures, in this case Viterbi algorithm is preferable. The decoding capability of HMM helps to find the sequence of the fault behavior whether it comes only from the stuck-at-fault and/or other fault behavior that generated the given signatures. In the FPGA fault emulation and radiation experiment we are interested to find the fault behavior of the system as they represent some valuable information that can later be used to give to the Simulink model to reproduce the signatures (if we know the fault behavior and their respective probabilities) as shown in Figure~\ref{fig:HMMsig}.

 
\textbf{Example:} Consider an example of the signature and stuck-at-fault; an FPGA test designer can only sense the signature but wants to know the fault behavior to make the high-level model based on this information.


\textbf{Answer}: We will use the \underline{Viterbi Algorithm} to determine the most probable sequence of fault behavior by giving the sequence of signatures and HMM. Inshort, decoding helps to find the hidden sequence most likely to have generated a sequence of observation --- solved using Viterbi algorithm as shown in Figure~\ref{fig:HMMsig-Vit}. 
\begin{figure}[tb!]
 \centering
  \captionsetup{justification=centering}    
   \includegraphics[scale=0.8]{Figures/HMM-plus-viterbi.pdf}
   \caption{Signatures to Fault model}
\label{fig:HMMsig-Vit}
\end{figure}
\subsubsection{Viterbi Algorithm}
The Viterbi algorithm is based on the assumption that the most \textit{likely} path (hidden states sequence), $Q^* = argmax_Q P(Q|O) $, is a good estimation of the sequence of hidden states that generated the observed sequence $(O)$. Viterbi algorithm generates a path $X = (x_1, x_2,...,x_T)$, which is a sequence of states $x_n \in S = {s_1, s_2,...,s_k}$ that produce the sequence of observations $Y = (y_1,y_2,...,y_T) \in {1,2,...,N}^T N$ $=$ $Observation$ $space$.


\textbf{Steps to apply Viterbi}
\begin{itemize}
\item Observation space $O =  {o_1,o_2,...,o_N}$
\item state space $S = {s_1,s_2,...,s_K}$
\item an array of the initial probabilities $\Pi = (\pi_1, \pi_2,...,\pi_K)$; $\pi_i$ $is$ $the$ $probability$ $the$ $x_1$ $==$ $s_i$
\item a sequence of observations $Y = (y_1,y_2,...,y_T)$; $yt == i$ observation at the time $t$ is $o_i$
\item the state transition matrix $A_ij$ stores the transition probability from state $s_i$ to $s_j$
\item emission matrix $B_ij$ stores the probability of observing $o_j$ from the state $s_i$
\end{itemize}  
\textbf{Recursion:}
We need to determine the hidden states by calculating $P(X|O)$. However, this brute force approach becomes intractable if the number of state gets larger, as the number of state path grows exponentially $(N^T)$. So, we need to calculate the $argmax_X P(X|O)$ . The most likely sequence is given by the recursion relations.
\begin{center}
\begin{itemize}
\item $V_1,k = P (y_1|k) \times \pi_k$
\item $V_t,k = max_{x \in S} (P (y_t|k) \times a_x,k \times V_t-1,x)$
\item $V_t,k$ probability of the most probably sequence $P (x_1, x_2,...,x_T, y_1,y_2,...,y_T)$
\end{itemize}
\end{center}
\textbf{Final:}
The most likely hidden state sequence $X = (x_1,x_2,...,x_N)$ 
\subsection{Learning}
The third application of the HMM is optimizing the parameters of the model. There are two possible solutions to this problem either to choose supervised learning and unsupervised learning.  In supervised learning, you have one input variable, one output variable, and use the algorithm to learn the mapping from the input to output, e.g., logistic regression, linear regression. If we  map our problem to supervised learning, in this case, our goal is to find the mapping function. In this case, whenever we have a new signature, the mapping function can predict the fault model. Now, consider a situation where only signature data is available and no information how these signatures come from the fault model.  This is the unsupervised learning because there is no answer available. In this thesis problem, we opt for the unsupervised learning, because we have only the data available that is a signature, and we will find the parameters that maximize the probability of the hidden sequence (fault models).  
We can also find the optimal solution for our model by using the learning application of the HMM. The learning application works on the model parameters and observations to find the model that fits the data. There are three different techniques to do: a) Maximum Likelihood Estimation, b) Viterbi Training, and c) Baum  Welch = Forward-Backward Algorithm. 
Suppose we have a HMM: $\Pi = (\pi, A, B)$. The Baum-Welch algorithm is used to find  a local maximum for $\Pi^* = arg max_{\Pi} P (O | \Pi)$, i.e., the HMM parameters $\Pi$ that maximize the probability of the observation.
The Baum-Welch works in the following way:
\begin{enumerate}
\item Find the forward probabilities with the forward algorithm.
\begin{itemize}
\item $\alpha_{i}(1) = \pi_{i} b_{i} (o_{1})$
\item $\alpha_{i}(t + 1) = b_{i}(o_{t+1}) \sum\limits^{N}_{j=1}\alpha_{j}(t)a_{ji}$
\end{itemize}
\item Find the backward probabilities with the backward algorithm.
\begin{itemize}
\item $\beta(t) = P(o_{t+1} o_{t+2},...,o_{T} | x_{t} = i, \Pi) $
\item $\beta_{i}(T) = 1$
\item $\beta_{i}(t) = \sum\limits^{N}_{j=1} a[x_i, x_j] b[x_j,o_{t + 1}\beta_{j}(t + 1)]$
\end{itemize}
\item Find the probability of state $i$ at time $t$.
\begin{itemize}
\item $P (x_t = i, O | \Pi) = P (o_1, o_2,...,o_t, x_t = i | \Pi) P (o_{t + 1}, o_{t + 2},...,o_{T} | x_{t} = i, \Pi) == \alpha_{i}(t) \beta_{i}(t) $
\item use the Bayes Theorem:
$P(x_t = i | O, \Pi) = \frac{P(x_t = i, O | \Pi)}{P(O | \Pi)} = \Upsilon_{i}(t)$
\end{itemize}
\item Find the probability of a transition from state $i$ to state $j$ at time $t$.
\begin{itemize}
\item $\xi_{t}(i,j) = P (x_t = i,x_{t + 1} = j | O, \Pi)$
\item These probabilities can be computed bu using the forward and backward variables:
$\xi_{t}(i,j) = \frac{\alpha_i (t) a[x_i x_j] b[x_j,o_{t + 1}] \beta_j (t + 1)   }{P(O | \Pi)}$
\end{itemize}
\item Find the expected transition and emission counts:
$\sum\limits_{t = 1}^{ T } \Upsilon_i (t)$ = expected number of transition from $x_i$
$\sum\limits_{t = 1}^{ T} \xi_t (i ,j)$ = expected number of transition from $x_i$ to $x_j$
\item Perform the parameter estimation by the ratio of expected count the maximization step:
$\overline{a}[x_i, x_j] = \frac{\sum\limits_{t = 1}^{T - 1}\xi_{t} (i, j)}{\sum\limits_{t = 1} T - 1 \Upsilon_{j}(t)} $

$\overline{b}[x_i, o_k] = \frac{\sum\limits_{t = 1}^{T - 1}\Upsilon_{j} (t) 1 (o_t = k)}{\sum\limits_{t = 1}^{T - 1} \Upsilon_{j}(t)} $
\item Stop the computation when the change in the log likelihood is smaller than a given threshold or the maximum iterations are passed.
\end{enumerate} 



\section{Faulty Components Library Utilization}
Once we have all the information about the HMM parameters, probabilities, fault models, a sequence of the hidden states. This information helps us to find the library of the faulty components. These components can be used by the designer to observe the faulty behavior of each sub-circuit on the system as shown in Figure~\ref{fig:lib1}.  If we want to find the faulty behavior of a MAC affected by the faulty behavior of an adder and/or an accumulator, we can observe this behavior by replacing the fault-free model with the faulty model.
 
\begin{figure}[tb!]
 \centering
  \captionsetup{justification=centering}    
   \includegraphics[scale=0.6]{Figures/MAC.pdf}
   \caption{Example of the faulty Component.}
\label{fig:lib1}
\end{figure}

We also have a plan to create a faulty finite state machines / faulty circuit models in Simulink. Then use the available functionality in the Simulink ( VHDL code generator) to find the respective VHDL entity, to get the individual VHDL entity as shown in Figure~\ref{fig:library1}. The VHDL entity helps to find the resources utilization of FPGA, and it is independent of the tool and technology easily portable to any FPGA as discussed in the Chapter~\ref{faultmodels} due to change in the tools some of the work became ineffective or out-of-date. If we have the faulty VHDL components, they are highly useful for understanding the system's behavior.


In short, we start developing the HMM from the Signatures. Then HMM helps to find the fault models associated with that particular circuit. We apply these fault models to an original circuit, make their faulty FSM / faulty circuit models, generate corresponding VHDL entity to measure the resource utilization.
%\textbf{Example:} A simulink designer needs to estimate the resource utilization of any digital circuit (operates in the radiation environment) that contain adder, multiplier, and flip-flops. He also interested to find the faulty behavior of this circuit. 
%
%A designer has a HMM which helps to find the possible fault models, based on this fault model example, designer will create the respective faulty circuit model / FSM. Once the designer have a FSM, he will create the VHDL entity and find the respective faulty circuit resource as well.tem's behavior.
%
\begin{figure}[tb!]
 \centering
  \captionsetup{justification=centering}    
   \includegraphics[scale=0.8]{Figures/library.pdf}
   \caption{VHDL entity creation and usage.}
\label{fig:library1}
\end{figure}

%
%\begin{figure}[tb!]
%
% \centering
%  \captionsetup{justification=centering}    
%   \includegraphics[scale=0.8]{Figures/simulink.pdf}
%   \caption{Library creation}
%\label{fig:lib}
%\end{figure}
%\begin{itemize}
The key utilization of this work:
\begin{itemize}
\item This work facilitates the fault simulation of faulty behavioral models. We put a fault-free circuit in Simulink based simulators and N fault models then it produces the N-faulty behavioral models.
\item We also produce the faulty VHDL entities which help to investigate the verification and testing.
\item This work able to analyze the multi-cycle error propagation behavior.
\item We will get the real-time fault occurrences and propagation data rather than user supposed fault probability distribution, e.g., normal, exponential, Poisson, Weibull, etc.
\item We will be able to study hardware faults at high-level of abstraction.
\item We will also be able to establish a relationship between the bit-flip information to the fault model.
\end{itemize}

\section{Project Plan}
\textbf{Summary} \\
\textbf{Phase  01:} The emulation platform will be the starting point of the research. We will use the sensitivity aware bit-flip technique for the configuration memory upsets. Selection of a suitable benchmark, which is probably ITC'99~\citep{ITC} used for the testing purpose and signature generation.\\
\textbf{Phase  02:} Evaluate the radiation-based experimental results for signature under the neutron radiation at Triumf.\\
\textbf{Phase 03:} Implement the above-mentioned methodology and high-level model for soft-error of the sequential circuits.


\chapter{Preliminary Results}


In this chapter, we present the preliminary results of this research work. These results are focused on the implementation of a probabilistically analysable instruction and data cache for the Ion MIPS32 processor on
FPGA. We developed a random placement and replacement policy that fulfills
all the requirements for PTA. Our experiments show that the cache fulfills all the requirements for PTA, and program timing can be determined with arbitrary accuracy. In addition, random placement and replacement improve the observed WCET from 6\% to 19\% w.r.t. a Least Recently Used policy.



\section{Relative Sensitivity Based Emulation}

This paper presents an FPGA implementation of a probabilistically analyzable
cache inspired by the simulation work presented
in~\cite{Kosmidis:2013:CDP:2485288.2485416}. In this paper we have
kept the same approach for the cache behaviour:
\begin{enumerate}
\item The cache uses a random replacement policy
\item The cache uses a parametric random placement policy based on
 a hash function
\item The cache placement is deterministic for each benchmark execution, but
  randomized across executions
\item We measure end-to-end execution time for a series of benchmarks
\end{enumerate}

\section{High-Level Fault Model}
\section{Sequential Circuits Fault model}
%\section{Hardware Implementation}
\label{Hardware Implementation}

%\subsection{Cache RTL Model}
\label{Cache Model}

We implemented an instruction and data cache for the Ion
MIPS32 processor~\cite{ION}. A completely novel, configurable cache
design was implemented in VHDL and integrated with the Ion core. The
cache is completely configurable (bus width, size, block size,
policies, etc.) with VHDL generics and could be easily ported to other
processor designs. Figure~\ref{fig:cache-structure} shows the main components of our design:

\begin{figure}
 \centering
  \captionsetup{justification=centering}    
   \includegraphics[scale=0.2]{figures/img/cache_structure_c.pdf}
   \caption{Structure of the proposed cache.}
\label{fig:cache-structure}
\end{figure}

\begin{enumerate}
\item The cache block contains the cache memory proper, as well as the
logic to manage the replacement policy (random and least-recently-used)
\item A hash function block that operates on the index signal to the
  cache, randomizing the mapping between memory blocks and cache blocks
\item A pseudo-random number generator (MT19937)
\item The Ion core, which provides a MIPS32 ISA and controls the whole
  system
\end{enumerate}

Our cache has three fundamentally novel features that enable probabilistic
timing analysis:
\begin{enumerate}
\item A random \textbf{placement} policy which uses a parametric hash
  function to shuffle the initial placement of blocks in the cache memory
\item A random \textbf{replacement} policy that uses high-quality random
numbers to provide statistically-verifiable guarantees that replacement
events are uniformly distributed among the available cache blocks
\item A high-quality pseudo-random number generation, with an extremely
long period, to generate random bits for the implementation of the cache
random policies
\end{enumerate}
%
%
%\subsection{Random Number Generation}

In our cache design, we used the Mersenne Twister algorithm to
generate random numbers. In particular, we used the MT19937 algorithm,
which is considered as a good hardware solution for a random number
generation~\cite{Matsumoto:1998:MTE:272991.272995}. MT19937 provides a
uniform pseudo number pattern with a period of
$2\textsuperscript{19937-1}$, with a width of 32 or 54 bits.  We used
the OpenCores implementation of MT19937~\cite{OpenCores}. The
synthesis report shows that the maximum clock frequency the design can
achieved is 147.016 MHz, with a throughput of 30 Msamples per second.

%\subsection{Parametric Hash Function}
%\label{PHF}
%
%The idea of using a parametric hash function for the implementation of
%random placement was given
%by~\cite{Kosmidis:2013:CDP:2485288.2485416}. This design is remodelled
%for this work, replacing their Multiply With Carry (MWC) random number
%generator with the MT19937, increasing the quality of the random
%numbers as well as the period.  The redesign was driven by the fact
%that MWC does not pass some statistical normality
%tests~\cite{bandyopadhyay2015discrete}, and its period might be
%insufficient for long running
%applications~\cite{Goresky:2003:EMR:945511.945514}.
%
%Standard placement assigns sets to cache lines based on the index bits
%of the memory address. If the placement policy assigns two memory
%addresses to the same cache set, they will systematically be in conflict. 
%To deal with this deterministic nature while randomizing the timing
%behaviour of the placement policy, we use a parametric hash
%function with a random number as an input.  A random number
%provides a unique and constant cache set mapping for each address.
%If the random number changes, the cache set in which the address is mapped changes. By changing random number only at a new execution, programs can be analyzed with end-to-end runs assuming that the cache is initially empty.
%Figure~\ref{fig:hash_function} shows the structure of the hash function.
%
%








%\subsection{Random Number Generation}
%
%In our cache design, we used the Mersenne Twister algorithm to
%generate random numbers. In particular, we used the MT19937 algorithm,
%which is considered as a good hardware solution for a random number
%generation~\cite{Matsumoto:1998:MTE:272991.272995}. MT19937 provides a
%uniform pseudo number pattern with a period of
%$2\textsuperscript{19937-1}$, with a width of 32 or 54 bits.  We used
%the OpenCores implementation of MT19937~\cite{OpenCores}. The
%synthesis report shows that the maximum clock frequency the design can
%achieved is 147.016 MHz, with a throughput of 30 Msamples per second.

%\subsection{Parametric Hash Function}
\label{PHF}


\begin{figure}[t!]

 \centering
  \captionsetup{justification=centering}    
   \includegraphics[scale=0.8]{figures/img/hash_function.pdf}
   \caption{The hash function uses a random number, the address bits, and four XOR stages to produce a random placement.}
\label{fig:hash_function}
\end{figure}



%\begin{figure}[t!]
%\includegraphics[width=\columnwidth]{figures/img/hash_function.pdf}
%\caption{The hash function uses a random number,
%the address bits, and four XOR stages to produce a random placement}
%\label{fig:hash_function}
%\end{figure}

The idea of using a parametric hash function for the implementation of
random placement was given
by~\cite{Kosmidis:2013:CDP:2485288.2485416}. This design is remodelled
for this work, replacing their Multiply With Carry (MWC) random number
generator with the MT19937, increasing the quality of the random
numbers as well as the period.  The redesign was driven by the fact
that MWC does not pass some statistical normality
tests~\cite{bandyopadhyay2015discrete}, and its period might be
insufficient for long running
applications~\cite{Goresky:2003:EMR:945511.945514}.

Standard placement assigns sets to cache lines based on the index bits
of the memory address. If the placement policy assigns two memory
addresses to the same cache set, they will systematically be in conflict. 
To deal with this deterministic nature while randomizing the timing
behaviour of the placement policy, we use a parametric hash
function with a random number as an input.  A random number
provides a unique and constant cache set mapping for each address.
If the random number changes, the cache set in which the address is mapped changes. By changing random number only at a new execution, programs can be analyzed with end-to-end runs assuming that the cache is initially empty.
Figure~\ref{fig:hash_function} shows the structure of the hash function.






%



%\section{Experimental Results}

\begin{table}[t]
\caption{Resource utilization and overhead (Virtex-5)}
\label{resource_table}
\begin{center}
\begin{tabular}{llllll}
\toprule
              & LRU & \multicolumn{3}{c}{RND} & Overhead \\
\cmidrule(lr){3-5}
              &   &  Cache & Hash	& MT19937 &  \\
\midrule
%Available     & 69120    &       69120   & 69120           \\
LUT Flip Flop  & 1904    &  1792   &  656   &  117  & 34.7\% \\ 
Slice LUTs     & 6026    &  5637   &  660   &  419  & 11.6\%  \\
%BRAMs          & 4       &  6      &  2     &  1    & 225\%\\
\bottomrule
\end{tabular}
\end{center}
\end{table}


\begin{figure}[t!]

 \centering
  \captionsetup{justification=centering}    
   \includegraphics[scale=0.8]{figures/img/boxplot.pdf}
   \caption{Execution Time Measurement}
\label{fig:boxplot}
\end{figure}
The architecture used in our experiments is the OpenCores Ion MIPS32
processor. We integrated both an I-cache and a D-cache, and we
implemented the whole system on the Xilinx ML505 FPGA evaluation
board, using the XC5VLX110T chip using  Xilinx ISE-14.4 and ModelSim 10.1.a. We used two separate 4~kB cache
memories for data and instructions, both with a 32-byte line size.  To
evaluate our design, we used M\"alardalen real-time benchmark~\cite{mrtc:bench} suite. We selected six benchmarks: \textit{cnt, bs,
  fac, crc, qsort-exam and select}. These benchmarks use arrays
and matrices, and have nested loops structures which are ideal to test
our design~\cite{Competitive}. We omitted those benchmarks using
external libraries and unstructured code to simplify the
software implementation and data collection. 



Each benchmark was run on multiple cache configurations profiles, and
we derived its execution time profile using MBPTA, with 30 runs per
profile to approximate a normal error distribution.
To show that our cache generates identically distributed execution times (as
required for PTA), we used the Kolmogorov-Smirnov test~\cite{books/daglib/0020904},
which shows that the null hypothesis (the data are normally distributed) cannot
be rejected for all benchmarks at the 5\% confidence level ($p>0.062$).
We compared our results (RND) with a standard Least-Recently-Used (LRU)
cache policy implementation.

Figures \ref{fig:boxplot} show the timing
distributions for all benchmarks on our cache from direct-mapped to
8-way associative, respectively. As an added advantage, our random
cache shows a 19\% improvement in worst case execution time w.r.t to
LRU for a direct-mapped cache, 11\% for 2-way cache, 8\% for a 4-way
cache, and 6\% for an 8-way cache. As expected, LRU gets closer to RND
as the number of ways increases: the number of conflict miss is
greatly reduced by additional ways.

\section{Conclusion}
In this paper, we present the RTL model of a randomized L1 data and
instruction cache. This cache uses a high-quality random number
generator for random placement and replacement. Random placement is
obtained with a parametric hash function that shuffles the association
between memory addresses and cache blocks. The cache is integrated
with the Ion MIPS32 processor, and verified to generate independent
and identically distributed timing events, such that Measurement-Based
Probabilistic Timing Analysis is possible (MBPTA). We test our cache
and MBPTA approach on a variety of benchmark from the M\"alardalen
benchmark suite and show a noticeable improvement (5-15\%) in terms of
measured Worst Case Execution Time (WCET) as well as enabling the
identification of safe probabilistic WCET (pWCET) bounds.  Future work
will consider the implementation of shared randomized caches for
multi-core architectures.

%
%\chapter{Conclusion}
% 

In this proposal, we have demonstrated our research will be focused on investigation of a design, methodologies, and implementation of a time predictable fault tolerant computing system evaluated by the probabilistic timing analysis (PTA). As our target domain is a real-time industry. We will stress the importance of a worst case execution time (WCET) estimation.
Nowadays, the investigations of new timing analysis techniques are an unavoidable need because of the growing complexity of a modern embedded computers and the aerospace industry will especially a benefit from the introduction of a such technologies in terms of  reliability and design costs. Our approach will leverage probabilistic approach to enable the timing analysis in computing systems.
As a consequence, this research has a potential to make computing systems smarter, more reliable, and easier to design and to program. At the same time, we think that our results will make decisive steps ahead in a fairly unexplored research area - integration of fault-tolerance techniques in time predictable computer architecture.


In our preliminary results, we showed that how probabilistically analysable cache can be integrated in a MIPS processor to make the foundation for a probabilistically analysable computing systems. The research published in~\cite{NEWCAS} proved the effectiveness of a measurement based probabilistic timing analysis (MBPTA). Furthermore, our most recent results demonstrated that probabilistic timing technique is a promising approach for the future timing analysis of a real-time aerospace embedded system.

Through this research, we hope to be able to have an impact on how computer engineers and system designers will think of probabilistic computing in near future, and contribute to create the next generation of a real-time embedded systems for aerospace industry.
We will target field-specific international  journals such as the ``ACM Transactions on Real-time  Systems'' and the ``ACM Transactions on Reconfigurable Technology and Systems'', or conferences including tracks dedicated to the automated design of embedded systems, such as the ``Design Automation Conference'' and the ``Design, Automation \& Test in Europe'' conference.

\section{Work Breakdown Structure}
Figure~\ref{wbs} presents the work breakdown structure (WBS) of our research project. At level 2 of this tree chart, we identify four groups of tasks: the ones related to the acquisition of knowledge; those related to the development of new knowledge; an experimental phase; and, finally, project management tasks.

Knowledge acquisition includes the class work done towards the credit requirements of the PhD program and the review of the scientific literature. 

The knowledge extension task can be split along the four research axes defined in Chapter~\ref{sec:approach}. The experimental phase goes from the definition of a test plan to the experimental evaluation of our prototype on a CubeSat  platform.

Project management tasks involve the writing of conference and journal articles, as well as the preparation of a thesis and its defense.

\begin{figure}[h]
\vspace{2cm}
\centering
\begin{tikzpicture}[
  level 1/.style={sibling distance=40mm},
  edge from parent/.style={->,draw},
  >=latex]

% root of the the initial tree, level 1
\node[root, fill = black!30] {Research Project}
% The first level, as children of the initial tree
  child {node[level 2,fill = black!10] (c1) {Knowledge Acquisition}}
  child {node[level 2,fill = black!10] (c2) {Knowledge Extension}}
  child {node[ level 2,fill = black!10] (c3) {Experimental Phase}}
  child {node[level 2,fill = black!10] (c4) {Project\\ Management}};

% The second level, relatively positioned nodes
\begin{scope}[every node/.style={level 3}]
\node [rounded corners,fill = white, below of = c1, xshift=15pt] (c11) {\footnotesize Classes};
\node [rounded corners,fill = white, below of = c11] (c12) {\footnotesize Literature\\ Review};

\node [rounded corners,fill = white, below of = c2, xshift=15pt] (c21) {\footnotesize PTA};
\node [rounded corners,fill = white, below of = c21] (c22) {{\footnotesize Predictable computing}};
\node [rounded corners,fill = white, below of = c22] (c23) {\footnotesize Fault tolerance};
\node [rounded corners,fill = white, below of = c23] (c24) {\footnotesize Reconfiguration.};
\node [rounded corners,fill = white, below of = c24] (c25) {\footnotesize Multicore Architecture.};

\node [rounded corners,fill = white, below of = c3, xshift=15pt] (c31) {\footnotesize Design};
\node [rounded corners,fill = white, below of = c31] (c32) {\footnotesize Algorithm Eval.};
\node [rounded corners,fill = white, below of = c32] (c33) {\footnotesize FPGA Implementation.};
\node [rounded corners,fill = white, below of = c33] (c34) {\footnotesize Tech. Demo.};

\node [rounded corners,fill = white, below of = c4, xshift=15pt] (c41) {\footnotesize Conferences};
\node [rounded corners,fill = white, below of = c41] (c42) {\footnotesize Journals};
\node [rounded corners,fill = white, below of = c42] (c43) { \footnotesize Thesis and\\ Graduation};
%\node [fill = black!30, below of = c43] (c44) {Thesis and\\ Graduation};

\end{scope}

% lines from each level 1 node to every one of its "children"
\foreach \value in {1,2}
  \draw[->] (c1.195) |- (c1\value.west);

\foreach \value in {1,...,5}
  \draw[->] (c2.195) |- (c2\value.west);

\foreach \value in {1,...,4}
  \draw[->] (c3.195) |- (c3\value.west);

\foreach \value in {1,...,3}
  \draw[->] (c4.195) |- (c4\value.west);
\end{tikzpicture}
\caption{Work breakdown structure.}
\label{wbs}
\end{figure}

\newpage
\section{Timetable}
The development of the tasks identified in Chapter~\ref{sec:approach}, and the most important milestones of this project are presented in Figure~\ref{timetable}.
In our intentions, the design of a time predictable computer architecture, the development of novel timing analysis techniques, and the FPGA prototypes implementation will unfold as a series of sequential tasks with relatively small interleaving. Dependability and real-time requirements, on the other hand, should be kept in mind throughout the whole advancement of the project.

\begin{figure}[h]
\centering
\begin{tikzpicture}
\begin{ganttchart}[
x unit=0.36cm,
y unit title=1.0cm,
y unit chart=1.5cm,
%vgrid,
hgrid,
inline,
]{1}{48}
\gantttitle{Research Project}{48} \\
\gantttitle{2015}{12} \gantttitle{2015}{12} \gantttitle{2016}{12} \gantttitle{2017}{12}\\



\ganttbar[bar height=.4]{Literature Review}{1}{16}\\
%\ganttmilestone[]{Comp. Exam}{20}\\
%\ganttmilestone[]{AHS}{6}
%\ganttmilestone[]{TODAES}{12}\\
\ganttbar[bar height=.4]{Leon 3 processor analysis}{13}{32} \\
\ganttbar[bar height=.4]{Architectural Modifications}{17}{32} \\
\ganttbar[bar height=.4]{Probabilistic component design}{18}{38} \\
%\ganttmilestone[]{TAAS}{30}\\
\ganttbar[bar height=.4]{Novel Timing analysis techniques}{22}{38} \\
%\ganttmilestone[]{DAC}{36}\\
\ganttbar[bar height=.4]{FPGA Implementation}{27}{38} \\
%\ganttmilestone[]{TRETS}{40}\\
\ganttbar[bar height=.4]{Tech. Demo.}{35}{42} \\
%\ganttmilestone[]{IAC}{44}\\
\ganttbar[bar height=.4]{Thesis Writing}{35}{46}
%\ganttmilestone[]{Thesis}{44}\\
%\ganttbar[bar height=.4]{The \emph{PolyOrbite} Project}{1}{44} 

%\ganttlink{elem0}{elem1}
%\ganttlink{elem0}{elem2}
%\ganttlink{elem0}{elem3}

%\ganttlink{elem3}{elem4}

%\ganttlink{elem5}{elem6}
%\ganttlink{elem5}{elem7}

%\ganttlink{elem7}{elem8}
\end{ganttchart}
\end{tikzpicture}
\caption{Timetable.}
\label{timetable}
\end{figure}





%%%
%%%  SYNTHESE DES TRAVAUX
%%%
%\section{Synthèse des travaux}
%Texte.
%
%%%
%%%  LIMITATIONS
%%%
%\section{Limitations de la solution proposée}\label{sec:Limitations}
%
%%%
%%%  AMELIORATIONS FUTURES
%%%
%\section{Améliorations futures}
%Texte.



%\section{Layout tests}

%In this sections, several environments are presented.
%
%\subsection{Listing tests}
%
%Presentation of the main listing environments: enumerations and lists.
%
%\subsubsection{Enumerations: enum environment}
%
%Enum environment test:
%\begin{enumerate}
% \item test 1
% \item test 2
%\end{enumerate}
%
%\subsubsection{Lists: itemize environment}
%
%Test of the itemize environment
%\begin{itemize}
% \item test 1
% \item test 2
%\end{itemize}
%
%\subsection{Equations tests}
%
%Layout of the equations:
%
%\begin{equation}
%   \beta = 8
%\end{equation}
%
%\begin{equation}
%   \gamma = \alpha \times 3
%\end{equation}
%
%\section{Second section}
%
%Example of a second section, to test the layout in the table of contents.
%
%
%%%- Second demo chapter -%%
%\chapter{Second Chapter}
%
%\section{Table layout tests}
%
%Tables have the same constraints than the figures, except for the caption that has to be on top.
%
%
%\begin{table}
%		\parbox{0.65\textwidth}{\caption{Test of a long table caption, with linebreak.}} % Contrainte manuelle de la largeur de la légende
%		\begin{tabular}{|c|c|c|c|c|c|c|c|}
%		\hline
%			{\bf titre} & {\bf titre} & {\bf titre} & {\bf titre} & {\bf titre} & {\bf titre} & {\bf titre} & {\bf titre} \\
%	  \hline
%			blá & blá & blá & blá & blá & blá & blá & blá \\
%	  \hline
%			blá & blá & blá & blá & blá & blá & blá & blá \\
%	  \hline
%			blá & blá & blá & blá & blá & blá & blá & blá \\
%	  \hline
%			blá & blá & blá & blá & blá & blá & blá & blá \\
%	  \hline
%			blá & blá & blá & blá & blá & blá & blá & blá \\
%	  \hline
%			blá & blá & blá & blá & blá & blá & blá & blá \\
%	  \hline
%		\end{tabular}
%\end{table}
%
%
%\section{References test}
%
%\subsection{References to the bibliography}
%
%Reference from the bibliography \cite{BookExample}.
%
%\subsection{References to the list of references "refs"}
%
%References from the list of references "refs", declared at the beginning of the document \citerefs{Test}.
%
%\subsection{References to a label of the document}
%
%Reference to a Figure associated to a label: Figure \ref{fig:vueEts}.
%
%\subsection{URL references}
%
%\subsubsection{Test of "href"}
%
%Href is used to integrate a link to a text:
%\href{http://www.etsmtl.ca/Etudiants-actuels/Cycles-sup/Realisation-etudes/Guides-gabarits}{Link to the template page.}.
%
%\subsubsection{Test de url}
%
%Url is used to format a clickable link:
%\url{http://www.etsmtl.ca/Etudiants-actuels/Cycles-sup/Realisation-etudes/Guides-gabarits}.
%
%%%- Third demo chapter -%%
%\chapter{Example of a thesis by article, with integrated article}
%
%% \articleAuthors{Names}{Affiliations} is used to print authors and their affiliations. Names must be declared as {First name1 Last Name1}{First Name2 Last Name2}... , as well as the affiliations
%%Exemple \articleAuthors{{First Name Last Name}}{{Affiliation}}
%
%\articleAuthors{
%{First name Last name\textsuperscript{1}}{First name Last name\textsuperscript{1}}
%}{
%{\textsuperscript{1} Département de Génie Mécanique, École de Technologie Supérieure,\\
%1100 Notre-Dame Ouest, Montréal, Québec, Canada H3C 1K3
%\\~\\
%Article soumis à la revue « Vecteur environnement » en septembre 2010.}
%}
%
%\section{Section 1}
%
%\lipsum[1] % Text filling, to have an example of the layout
%
%%%- Conclusion -%%
%\begin{conclusion}
%
%\lipsum[1] % Text filling, to have an example of the layout
%
%\end{conclusion}
%



%%%%%%%%%%%%%%%%%%%%%%%%%%%%%%%%%%%%%%%%%%%%%%%%%%%%
%%  Appendix example:
%%%%%%%%%%%%%%%%%%%%%%%%%%%%%%%%%%%%%%%%%%%%%%%%%%%%
%\appendix
%
%%% To use more than one appendix
%\multiannexe
%
%%% Appendix from an external file
%% \include{extApp}

\chapter{Appendix example}


\section{First section of the appendix}


\subsection{Figures in annexes}

%\begin{figure}
%	\centering
%	\fbox{
%		\includegraphics[width=0.75\textwidth]{Figures/vueEts.jpg}
%	}
%	 \\ \parbox{0.75\textwidth}{\caption{Figure in an appendix.}\label{fig:testAp}}
%\end{figure}
%
%\begin{figure}
%	\centering
%	\fbox{
%	\parbox{0.975\linewidth}{
%	\subfloat[first figure]{
%		\includegraphics[width=0.47\linewidth]{Figures/vueEts.jpg}
%	}\hspace{0.009\linewidth}
%	\subfloat[second figure]{
%		\includegraphics[width=0.47\linewidth]{Figures/vueEts.jpg}
%	}
%	
%	\subfloat[third figure]{
%		\includegraphics[width=0.47\linewidth]{Figures/vueEts.jpg}
%	}\hspace{0.009\linewidth}
%	\subfloat[fourth figure]{
%		\includegraphics[width=0.47\linewidth]{Figures/vueEts.jpg}
%	}}}
%	\\ \parbox{0.975\linewidth}{\caption{Subfig example.}}
%\end{figure}
%
%In the annexes, the figures are declared in the same way. Their numbering changes automatically (e.g. Figure \ref{fig:testAp}).
%
%\subsubsection{Tables in annexes}
%
%\begin{table}
%		\parbox{0.65\textwidth}{\caption{Table in an appendix.}\label{tab:testAp}}
%
%		\begin{tabular}{|c|c|c|c|c|c|c|c|}
%		\hline
%			{\bf titre} & {\bf titre} & {\bf titre} & {\bf titre} & {\bf titre} & {\bf titre} & {\bf titre} & {\bf titre} \\
%	  \hline
%			blá & blá & blá & blá & blá & blá & blá & blá \\
%	  \hline
%			blá & blá & blá & blá & blá & blá & blá & blá \\
%	  \hline
%			blá & blá & blá & blá & blá & blá & blá & blá \\
%	  \hline
%			blá & blá & blá & blá & blá & blá & blá & blá \\
%	  \hline
%			blá & blá & blá & blá & blá & blá & blá & blá \\
%	  \hline
%			blá & blá & blá & blá & blá & blá & blá & blá \\
%	  \hline
%		\end{tabular}
%\end{table}
%
%Same behaviour for the tables (e.g., Table \ref{tab:testAp}).


%%%%%%%%%%%%%%%%%%%%%%%%%%%%%%%%%%%%%%%%%%%%%%%%%%%
% BIBLIOGRAPHY AND REFERENCES
%%%%%%%%%%%%%%%%%%%%%%%%%%%%%%%%%%%%%%%%%%%%%%%%%%%

%%- Bibliography -%%
\newpage
% Single spacing for the bibliography
\begin{spacing}{1}
	\nocite{*} % The "nocite" command can be used to print references that haven't been used in the document. The "*" option specifies that every reference should be printed
	\bibliographystyle{bibETS} % ETS bibliography style
	\addcontentsline{toc}{chapter}{BIBLIOGRAPHY} % Addition of the bibliography in the table of contents

	\bibliography{biblio_en} % List of bibliography files, biblio.bib is an example

\end{spacing}

%%- Other list of references, "refs" example --%
%%%%%%%%%%%%%%%%%%%%%%%%%%%%%%%%%%%%%%%%%%%%%%%%%%%
% IMPORTANT: HOW TO COMPILE AND PRINT ADDITIONAL REFERENCES (replace "refs" by the chosen name)
%%%%%%%%%%%%%%%%%%%%%%%%%%%%%%%%%%%%%%%%%%%%%%%%%%%
% Follow these three steps:
%   1. Compile the document once, to save the used references in refs.aux
%   2. Compile the references
% 		- On Linux: Use the "bibtex refs" command in the document folder
%		- On MacOSX (MacTex distribution): Use the "/usr/texbin/bibtex refs" command in the document folder
%		- On Windows: Edit the "update_refs.bat" script to put the right suffix ("refs" here), and launch the script
%   3. Recompile the document TWICE
%%%%%%%%%%%%%%%%%%%%%%%%%%%%%%%%%%%%%%%%%%%%%%%%%%%

\newpage
% Same commands than for the bibliography, only with the "refs" suffix
\begin{spacing}{1}
	%\nociterefs{*}
	\bibliographystylerefs{bibETS}
	\addcontentsline{toc}{chapter}{LIST OF REFERENCES}

	\bibliographyrefs{refs}

\end{spacing}

\end{document}
